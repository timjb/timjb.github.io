\documentclass{cheat-sheet}

\pdfinfo{
  /Title (Zusammenfassung Stochastik 1)
  /Author (Tim Baumann)
}

\newcommand{\Alg}{\mathfrak{A}}
\newcommand{\LebAlg}{\mathfrak{L}} % Lebesgue-Borel-Mengen
\renewcommand{\P}{\mathbb{P}}
\newcommand{\PD}[2]{\P(#1\,|\,#2)}

\begin{document}

\maketitle{Zusammenfassung Stochastik \rom{1}}

\section{Der abstrakte Maßbegriff}

% TODO: Ereignisalgebra?

\begin{defn}
  Eine \emph{Ereignisalgebra} oder \emph{Boolesche Algebra} ist eine Menge $\mathfrak{A}$ mit zweistelligen Verknüpfungen $\wedge$ (\glqq und\grqq) und $\vee$ (\glqq oder\grqq), einer einstelligen Verknüpfung $\overline{\,\cdot\,}$ (Komplement) und ausgezeichneten Elementen $U \in \mathfrak{A}$ (unmögliches Ereignis) und $S \in \mathfrak{A}$ (sicheres Ereignis), sodass für $A, B, C \in \mathfrak{A}$ gilt:

  \begin{multicols}{2}
    \scriptsize
    \begin{enumerate}[label=\roman*.,leftmargin=2em]
      \item $A \wedge A = A$
      \item $A \wedge B = B \wedge A$
      \item $A \wedge S = A$
      \item $A \wedge U = U$
      \item $A \wedge \overline{A} = U$
      \item $A \wedge (B \wedge C) = (A \wedge B) \wedge C$
      \item $A \vee A = A$
      \item $A \vee S = S$
      \item $A \vee U = A$
      \item $A \vee \overline{A} = S$
      \item $A \vee (B \vee C) = (A \vee B) \vee C$
      \item $A \wedge (B \vee C) = (A \wedge B) \vee (A \wedge C)$
    \end{enumerate}
  \end{multicols}
\end{defn}

\begin{defn}
  Sei $\mathfrak{A}$ eine Boolesche Algebra. Dann definiert
  \[ A \leq B \colon\iff A \wedge B = B \]
  eine Partialordnung auf $\mathfrak{A}$, gesprochen $A$ impliziert $B$.
\end{defn}

\begin{defn}
  Eine \emph{Algebra} (auch Mengenalgebra) $\mathfrak{A} \subset \mathcal{P}(\Omega)$ ist ein System von Teilmengen einer Menge $\Omega$ mit $\emptyset \in \mathfrak{A}$, das unter folgenden Operationen stabil ist:
  \begin{itemize}
    \item Vereinigung: $A, B \in \mathfrak{A} \implies A \cup B \in \mathfrak{A}$
    \item Durchschnitt: $A, B \in \mathfrak{A} \implies A \cap B \in \mathfrak{A}$
    \item Komplementbildung: $A \in \mathfrak{A} \implies A^c \coloneqq \Omega \backslash A \in \mathfrak{A}$
  \end{itemize}
\end{defn}

\begin{satz}[Isomorphiesatz von Stone]
Zu jeder Booleschen Algebra $\mathfrak{A}$ gibt es eine Menge $\Omega$ derart, dass $\mathfrak{A}$ isomorph zu einer Mengenalgebra $\mathfrak{A}$ in $\mathcal{P}(\Omega)$ ist.
\end{satz}

\begin{defn}
  Eine \emph{$\sigma$-Algebra} ist eine Algebra $\mathfrak{A} \subset \mathcal{P}(\Omega)$, die nicht nur unter endlichen, sondern sogar unter abzählbaren Vereinigungen stabil ist, d.\,h.

  \[ (A_n)_{n \in \N} \text{ Folge in } \mathfrak{A} \implies \bigcup_{n = 0}^{\infty} A_n \in \mathfrak{A}. \]
\end{defn}

\begin{bem}
  Es gilt damit:

  \begin{itemize}
    \item $\Omega = \emptyset^c \in \mathfrak{A}$
    \item Abgeschlossenheit unter abzählbaren Schnitten:
  \[ (A_n)_{n \in \N} \text{ Folge in } \mathfrak{A} \implies \bigcap_{n = 0}^{\infty} A_n = \left( \bigcup_{n = 0}^{\infty} (A_n)^c \right)^c \in \mathfrak{A}. \]
  \end{itemize}
\end{bem}

\begin{defn}
  Sei $(A_n)_{n \in \N}$ eine Folge in einer $\sigma$-Algebra $\mathfrak{A}$. Dann sind der Limes Superior und Limes Inferior der Folge $A_n$ wie folgt definiert:

  \[ \limsup_{n \to \infty} A_n \coloneqq \bigcap_{n = 1}^{\infty} \bigcup_{m = n}^{\infty} A_n \in \mathfrak{A} \]
  \[ \liminf_{n \to \infty} A_n \coloneqq \bigcup_{n = 1}^{\infty} \bigcap_{m = n}^{\infty} A_n \in \mathfrak{A} \]
\end{defn}

\begin{bem}
  In einer $\sigma$-Algebra, in der die Mengen mögliche Ereignisse beschreiben, ist der Limes Superior das Ereignis, das eintritt, wenn unendlich viele Ereignisse der Folge $A_n$ eintreten. Der Limes Infinum tritt genau dann ein, wenn alle bis auf endlich viele Ereignisse der Folge $A_n$ eintreten.
\end{bem}

\begin{defn}
  Ein \emph{Ring} $\mathfrak{A} \subset \mathcal{P}(\Omega)$ ist ein System von Teilmengen einer Menge $\Omega$ mit $\emptyset \in \mathfrak{A}$, das unter folgenden Operation stabil ist:

  \begin{itemize}
    \item Vereinigung: $A, B \in \mathfrak{A} \implies A \cup B \in \mathfrak{A}$
    \item Differenz: $A, B \in \mathfrak{A} \implies B \backslash A = B \cap A^c \in \mathfrak{A}$
  \end{itemize}

  Ein Ring, der nicht nur unter endlicher, sondern sogar unter abzählbarer Vereinigung stabil ist, heißt \emph{$\sigma$-Ring}.
\end{defn}

\begin{bem}
  $\mathfrak{A}$ ($\sigma$-)\,Algebra $\iff$ $\mathfrak{A}$ ($\sigma$-)\,Ring und $\Omega \in \mathfrak{A}$.
\end{bem}

\begin{satz}
  Sei $(\mathfrak{A}_i)_{(i \in I)}$ eine Familie von ($\sigma$-)\,Ringen / ($\sigma$-)\,Algebren über einer Menge $\Omega$. Dann ist auch $\cup_{i \in I} \mathfrak{A}_i$ ein ($\sigma$-)\,Ring / eine ($\sigma$-)\,Algebra über $\Omega$.
\end{satz}

%\begin{defn}
%  Sei $$
%\end{defn}

% ...

\begin{satz}
  Sei $A_1, A_2, ...$ eine Zerlegung von $\Omega$ und $B \in \Alg$. Dann gilt
  \begin{align*}
    \P(B) &= \sum_{n=1}^\infty \PD{B}{A_n} \P(A_n) \quad \text{(Formel der totalen Wkt.)}\\
    \PD{A_n}{B} &= \frac{\PD{B}{A_n} \cdot \P(A_n)}{\P(B)} \quad \text{(Formel von Bayes)}
  \end{align*}
\end{satz}

\begin{defn}
  Zwei Ereignisse $A, B \in \Alg$ heißen (stochastisch) ($\P$-)unabhängig, falls
  \[ \P(A \cap B) = \P(A) \cdot \P(B). \]
\end{defn}

\begin{satz}
  $A, B$ unabhängig $\iff$ $\PD{B}{A} = \P(B)$.
\end{satz}

\begin{defn}
  Eine Familie $A_i)_{i \in I} \subset \Alg$ ($I$ endlich, abzählbar oder überabzählbar) heißt \emph{vollständig unabhhängig}, falls
  \[ \P(A_{i_1} \cap A_{i_2} \cap ... \cap A_{i_m}) = \P(A_{i_1}) \cdot \P(A_{i_2}) \cdots \P(A_{i_n}) \]
  für beliebige $i_1, ..., i_n \in I, 2 \leq n < \infty$ und \emph{paarweise unabh.}, falls
  \[ \P(A_i \cap A_j) = \P(A_i) \cdot \P(A_j) \text{ für } i, j \in I, i \not= j. \]
\end{defn}

\begin{acht}
  Aus paarweiser Unabhängigkeit folgt nicht vollständige Unabhängigkeit (Gegenbeispiel: Bernsteins Tetraeder).
\end{acht}

\begin{defn}
  Sei $(\Omega, \Alg, \P)$ ein Wahrscheinlichkeitsraum und $\Alg_1, \Alg_2 \subset \Alg$ zwei Ereignissysteme. Dann sind $\Alg_1$ und $\Alg_2$ unabhängig, falls $\P(A_1 \cap A_2) = \P(A_1) \cdot \P(A_2)$ für alle $A_1 \in \Alg_1, A_2 \in \Alg_2$.
\end{defn}

\begin{satz}
  Seien $\Alg_1, \Alg_2$ zwei unabhängige Ereignisalgebren. Dann sind die $\sigma$-Algebren $\tilde{\Alg_1} = \sigma(\Alg_1)$ und $\tilde{\Alg_2} = \sigma(\Alg_2)$ unabhängig.
\end{satz}


\begin{satz}[von Lusin]
  $f : ([a, b], \LebAlg([a, b])) \to (\R^1, \LebAlg(\R^1))$ ist Borel-messbar $\iff$ $\forall \epsilon > 0 : \exists K\epsilon \subset [a, b]$ abgeschlossen mit $\lambda_1(\R^1 \setminus K_\epsilon)$ und $f|_{K_\epsilon}$ stetig.
\end{satz}

\begin{satz}
  Folgerung: Es sind messbar
  \begin{itemize}
    \item monotone Funktionen
    \item Funktionen mit endlicher Variation
    \item Càdlàg-Funktionen, das sind Funktionen $f : [a, b] \to \R$ mit $\lim_{\epsilon \downarrow 0} f(x+\epsilon) = f(x)$ für alle $x \in \left[a, b\right[$.
  \end{itemize}
\end{satz}

\begin{defn}
  Eine $\Alg$-messbare numerische Funktion $X$ über einem Wahrscheinlichkeitsraum $(\Omega, \Alg, \P)$ heißt \emph{Zufallsgrüße} oder \emph{Zufallsvariable}.
\end{defn}

\begin{defn}
  Die durch die ZG $X$ auf $(\R^1, \LebAlg(\R^1))$ induzierte Bildmaß $P_X$
  \[ P_X(B) = \P(X^{-1}(B)) = \P(\Set{ \omega \in \Omega  }{ X(\omega) \in B }) \]
  heißt \emph{Verteilung} der ZG $X$.
  \[ F_X(x) = P_X(\left]-\infty, x\right]) = \P(\Set{ \omega \in \Omega }{ X(\omega) \leq x }) \]
  heißt die \emph{Verteilungsfunktion} der ZG $X$.
\end{defn}

\begin{satz}
  $F$ sei eine VF auf $\R^1$. Dann existiert ein Wahrscheinlichkeits-Raum $(\Omega, \Alg, \P$ und eine ZG $X$ derart, dass
  \[ F_X(x) = F(x) \text{ für } x \in \R^1 \]
\end{satz}

\begin{nota}
  Sei $X$ eine Zufallsgröße und $B \in \LebAlg(\overline{\R}^1)$. Dann schreibe
  \[ \{ X \in B \} = X^{-1}(B). \]
\end{nota}

\begin{defn}
  Eine endliche Familie von Zufallsgrößen $X_1, ..., X_n$ heißt \emph{stochastisch unabhängig}, falls
  \[ \P(\bigcap_{i=1}^n \{ X_i \in B_i \}) = \prod_{i=1}^n \P(\{ X_i \in B_i \}) \text{ für alle } B_i \in \mathcal{L}(\overline{R}^1), i = 1, ..., n. \]
\end{defn}

\begin{satz}
  Seien $X_1, ..., X_n$ unabhängige Zufallsgrößen über $(\Omega, \Alg, \P)$ von $g_1, ..., g_n$ Borel-messbare Funktionen von $\R^1$ nach $\R^1$. Dann sind auch die Zufallsgrößen $Y_i \coloneqq g_i \circ X_i$ unabhängig über $(\Omega, \Alg, \P)$.
\end{satz}

% Definition: einfache Funktionen
% Definition: $\mu$-Integral für einfache Funktionen

% Eigenschaften: Linearität, Monotonie, \Int{\Omega}{}{\ind_A}{\mu} = \mu(A)

\begin{satz}
  Sei $0 \leq f_1 \leq f_2 \leq ...$ eine isotone Folge elementarer Funktionen über $(\Omega, \Alg)$. Dann gilt für jede elementare Funktion $f$ mit $f \leq \sup_{n \in N} f_n$ die Ungleichung $\Int{\Omega}{}{f}{\mu} \leq \sup_{n \in \N} \Int{\Omega}{}{f_n}{\mu}$.
\end{satz}

% Folgerung
\begin{satz}
  Seien $(f_n)_{n \in \N}$ und $(g_n)_{n \in \N}$ isotone Folgen elementarer Funktionen mit $\sup_{n \in \N} f_n = \sup_{n \in \N} g_n$. Dann ist $\sup_{n \in \N} \Int{\Omega}{}{f_n}{\mu} = \sup_{n \in \N} \Int{\Omega}{}{g_n}{\mu}$.
\end{satz}

% Satz: Jede nichtnegative, messbare, numerische Funktion ist Grenzwert einer Folge einfacher Funktionen
% Def: $\mu$-Integral von solchen Funktionen

% Integral von nicht unbedingt nicht-negativen messbaren Funktionen

\begin{satz}
  Sei $f : (\Omega, \Alg, \mu) \to (\overline{R}^1, \LebAlg(\R^1))$ sein $\Alg$-messbar, numerisch. Dann gilt:
  \begin{align*}
    & \text{$F$ ist $\mu$-integrierbar} \\
    \iff & \text{$f^+$ und $f^-$ sind $\mu$-integrierbar mit $\Int{\Omega}{}{f^{\pm}}{\mu} < \infty$} \\
    \iff & \Int{\Omega}{}{|f|}{\mu} < \infty \\
    \iff & \Int{\Omega}{}{g}{\mu} < \infty \text{ für eine $\Alg$-messbare, numerische Funktion mit $|f| \leq g$}.
  \end{align*}
\end{satz}

\begin{satz}
  Seien $f, g : (\Omega, \Alg, \mu) \to (\R^1, \LebAlg(\R^1))$ $\mu$-integrierbar. Dann sind $f \pm g$, $f \vee g$, $f \wedge g$ und $\alpha \cdot f$ für $\alpha \in \R^1$ $\mu$-integrierbar und es gilt
  \begin{align*}
    \Int{\Omega}{}{\alpha \cdot f + \beta \cdot g}{\mu} = \alpha \Int{\Omega}{}{f}{\mu} + \beta \Int{\Omega}{}{g}{\mu}, & \quad
    \left| \Int{\Omega}{}{f}{\mu} \right| \leq \Int{\Omega}{}{|f|}{\mu}, \\
    f \leq g \implies \Int{\Omega}{}{f}{\mu} &\leq \Int{\Omega}{}{g}{\mu}
  \end{align*}
\end{satz}

\begin{defn}
  Mit $L^p(\mu) = L^p(\Omega, \Alg, \mu)$ bezeichnen wir den normierten Vektorraum der aus den Funktionen $f : (\Omega, \Alg, \mu) \to (\R^1, \LebAlg(\R^1))$ mit $\Int{\Omega}{}{|f|^p}{\mu} < \infty$ für $1 \leq p \leq \infty$ besteht. Die Norm in diesem Raum wird durch
  \[ \|f\|_p \coloneqq \left( \Int{\Omega}{}{|f|^p}{\mu} \right)^{1/p} \]
  definiert. Es kann gezeigt werden, dass die Normeigenschaften erfüllt sind.
\end{defn}

% Definition: Konvergenz im $L^p(\mu)$

\begin{bem}
  Der $L^p(\mu)$ ist ein vollständiger normierter Raum, d.\,h. jede Cauchy-Folge bzgl. der Norm $\| \cdot \|_p$ ist auch konvergent. Im Spezialfall $p = 2$ heißt $L^p(\mu)$ Hilbertraum der quadratisch integrierbaren Funktionen mit Skalarprodukt $\langle f , g \rangle = \Int{\Omega}{}{f \cdot g}{\mu}$. Es gilt in diesem Fall außerdem die Cauchy-Schwarz-Bunjakowski-Ungleichung:
  \[ \| f \cdot g \|_{1} \leq \|f\|_2 \cdot \|g\|_2 \]
  Höldersche Ungleichung:
  \[ \| f \cdot g \|_{1} \leq \|f\|_p \cdot \|g\|_q \]
  wobei $p, q \in [1, \infty]$ mit $\tfrac{1}{p} + \tfrac{1}{q} = 1$.
\end{bem}

% Konvergenzsätze für Integrale messbarer Funktionen

% 3.5.1 Satz von der monotonen Konvergenz

\begin{satz}
  Sei $f_n : (\Omega, \Alg, \mu) \to (\R^1, \LebAlg(\R^1))$ $\Alg$-messbar und $0 \leq f_1 \leq f_2 \leq ...$. Dann gilt
  \[ \Int{\Omega}{}{\sup_{n \in \N} f_n}{\mu} = \sup_{n \in \N} \Int{\Omega}{}{f_n}{\mu} \]
\end{satz}

% Folgerung
\begin{satz}[von Beppo Levi]
  Sei $(f_n)_{n \in \N}$ eine Folge monotoner nichtnegativer, $\Alg$-messbarer, numerischer Funktionen auf $(\Omega, \Alg, \mu)$. Dann gilt:
  \[ \Int{\Omega}{}{\sum_{n=1}^{\infty} f_n}{\mu} = \sum_{n=1}^\infty \Int{\Omega}{}{f_n}{\mu} \]
\end{satz}

% Folgerung
\begin{satz}
  $f$ sei $\Alg$-messbar, nichtnegativ und $\mu$-integrierbar. Dann ist
  \[ \nu(A) \coloneqq \Int{A}{}{f}{\mu} = \Int{\Omega}{}{f \cdot \chi_A}{\mu} \]
  ein ednliches Maß auf $(\Omega, \Alg)$.
\end{satz}

\begin{satz}[Lemma von Fatou]
  Sei $f_n : (\Omega, \Alg, \mu) \to (\R^1, \LebAlg(\R^1))$ eine Folge $\Alg$-messbarer, nichtnegativer Funktionen. Dann gilt:
  \[ \Int{\Omega}{}{\liminf_{n \to \infty} f_n}{\mu} \leq \liminf_{n \to \infty} \Int{\Omega}{}{f_n}{\mu} \]
\end{satz}

\end{document}
