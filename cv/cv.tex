\documentclass[11pt,a4paper]{moderncv}

% moderncv themes
%\moderncvtheme[grey]{casual}                 % optional argument are 'blue' (default), 'orange', 'red', 'green', 'grey' and 'roman' (for roman fonts, instead of sans serif fonts)
\moderncvtheme[green]{classic}                 % optional argument are 'blue' (default), 'orange', 'red', 'green', 'grey' and 'roman' (for roman fonts, instead of sans serif fonts)
%\moderncvtheme[green]{classic}                % idem

% character encoding
\usepackage[utf8]{inputenc}
\usepackage[ngerman]{babel}
\usepackage[T1]{fontenc}
\usepackage{microtype}
\usepackage{fixltx2e}
\usepackage{lmodern}
\usepackage{fancyhdr}
\usepackage{ellipsis}
\usepackage{pdfpages}
\usepackage{graphicx}

%\newcommand{\url}{\href{#1}{#1}}

%\input{basis/metadata}

% adjust the page margins
\usepackage[scale=0.8]{geometry}
\setlength{\hintscolumnwidth}{4cm}            % if you want to change the width of the column with the dates
%\AtBeginDocument{\setlength{\maketitlenamewidth}{6cm}}  % only for the classic theme, if you want to change the width of your name placeholder (to leave more space for your address details
\AtBeginDocument{\recomputelengths \recomputecvlengths}                     % required when changes are made to page layout lengths

% personal data
\firstname{Tim}
\familyname{Baumann}
\title{}
\address{Lechallee 14d}{86399 Bobingen}
\mobile{0151\,--\,61490314}
\phone{08234\,--\,42232}
\email{tim@timbaumann.info}
\homepage{timbaumann.info}

%----------------------------------------------------------------------------------
%            content
%----------------------------------------------------------------------------------
\begin{document}
\maketitle

\section{Schulbildung}
\cventry{09/2005 -- 06/2012}{Gymnasium Königsbrunn}{Abschluss: Abitur (\O\,1,2)}{}{}{}  % arguments 3 to 6 can be left empty
\cventry{seit 10/2012}{Studium}{Universität Augsburg}{}{}{Studiengang: Mathematik B.\,Sc. (2. Semester)}

%\section{Praktika}
%\cventry{08/2011 -- 07/2012}{Informationstechnik}{Musterbetrieb}{Musterstadt}{}{}

\section{Sprachen}
\cvlanguage{Deutsch}{Muttersprache}{}
\cvlanguage{Englisch}{Gute Kenntnisse in Wort und Schrift}{}
\cvlanguage{Französisch}{Grundkenntnisse}{}

\section{Informationstechnische Kenntnisse}
%\cvcomputer{Betriebssysteme}{\textbf{Linux}, \textbf{Mac OS X}, Vim, Git u.~a.}{}{}
\cvline{Betriebssysteme}{\textbf{Linux}, \textbf{Mac OS X}}
\cvline{Tools}{\textbf{Vim}, \textbf{Git}}
\cvline{Programmiersprachen}{
  \small Sehr gute Kenntnisse in \textbf{Haskell}, \textbf{JavaScript}\newline{}
  Gute Kenntnisse in \textbf{HTML}, \textbf{CSS}, \textbf{SQL}, \textbf{Python}, \textbf{Agda}, \textbf{Coq}\newline{}
  Grundlagenkenntnisse von \textbf{C}, \textbf{Java}, \textbf{Ruby}}

\section{Open-Source-Projekte}

\cvline{ot.js/ot.hs/ot.lua}{Gemeinsames gleichzeitiges Bearbeiten von Dokumenten im Webbrowser mittels Operational Transformation.\newline{}\small\url{https://github.com/Operational-Transformation}}
\cvline{Substance.io}{Erweiterung der Exportfunktion um zahlreiche Formate; Refactoring des Benutzerinterfaces}

%\renewcommand{\listitemsymbol}{$\circ$} % change the symbol for lists

%\section{Sonstiges zur Informationstechnik}
%\cvline{Blog}{Führen eines Blogs über \LaTeX{}.}

%\section{Interessen}
%\cvline{}{Hobby}

\emptysection \closesection
\vspace*{2mm}
%\hspace*{4mm}\includegraphics[width=130px]{unterschrift.pdf}
%\newline{}
%Bobingen, den \today

\end{document}