\documentclass{cheat-sheet}

%\newcommand{\PS}{\mathcal{P}} % Powerset
%\newcommand{\PSO}{\PS(\Omega)} % Powerset
%\newcommand{\Alg}{\mathfrak{A}}
%\newcommand{\Ring}{\mathfrak{R}}
\newcommand{\K}{\mathbb{K}}
\newcommand{\supp}{\mathrm{supp}}

\pdfinfo{
  /Title (Zusammenfassung Funktionalanalysis)
  /Author (Tim Baumann)
}

\begin{document}

\maketitle{Zusammenfassung Funktionalanalysis}

\begin{nota}
  Sei im Folgenden $\K \in \{ \R, \C \}$.
\end{nota}

% TODO: Sei im Folgenden Y ein Banachraum?

\begin{defn}
  Ein \emph{Prä-Hilbertraum} ist ein $\K$-Vektorraum mit einem Skalarprodukt $\langle \cdot , \cdot \rangle$.
\end{defn}

\begin{defn}
  Sei $V$ ein $\K$-Vektorraum. Eine \emph{Fréchet-Metrik} ist eine Funktion $\rho : V \to \R_{\geq 0}$, sodass für $x, y \in V$ gilt:
  \begin{itemize}
    \item $\rho(x) = \rho(-x)$
    \item $\rho(x) = 0 \iff x = 0$
    \item $\rho(x + y) \leq \rho(x) + \rho(y)$
  \end{itemize}
\end{defn}

\begin{defn}
  Sei $(X, d)$ ein metrischer Raum und $A_1, A_2 \subset X$. Dann heißt
  \[ \mathrm{dist}(A_1, A_2) \coloneqq \inf \Set{ d(x,y) }{ x \in A_1, y \in A_2 } \]
  \emph{Abstand} zwischen $A_1$ und $A_2$.
\end{defn}

\begin{defn}
  Ein \emph{topologischer Raum} ist ein paar $(X, \tau)$, wobei $X$ eine Menge und $\tau \subset \mathcal{P}(X)$ ein System von offenen Mengen, sodass gilt:
  \begin{itemize}
    \item $\emptyset \in \tau$
    \item $\tilde\tau \subset \tau \implies \bigcup_{U \in \tilde\tau} U \in \tau$
    \item $U_1, U_2 \in \tau \implies U_1 \cap U_2 \in \tau$
  \end{itemize}
\end{defn}

\begin{defn}
  Ein topologischer Raum $(X, \tau)$ heißt \emph{Haussdorff-Raum}, wenn das Trennungsaxiom
  \[ \forall\, x_1, x_2 \in X : \exists\,U_1, U_2 \in \tau : x_1 \in U_1 \wedge x_2 \in U_2 \wedge U_1 \cap U_2 = \emptyset \]
  erfüllt ist.
\end{defn}

\begin{defn}
  Sei $(X, \tau)$ ein topologischer Raum. Eine Menge $A \subset X$ heißt abgeschlossen, falls $X \setminus A \in \tau$, also das Komplement offen ist.
\end{defn}

\begin{defn}
  Sei $(X, \tau)$ ein topologischer Raum und $A \subset X$. Dann heißen
  \begin{align*}
    A^{\circ} &\coloneqq \Set{ x \in X }{ \exists\,U \in \tau \text{ mit } x \in U \text{ und } U \subset A } \\
    \overline{A} &\coloneqq \Set{ x \in X }{ \forall\,U \in \tau \text{ mit } x \in U \text{ gilt } U \cap A \not= \emptyset }
  \end{align*}
  \emph{Abschluss} bzw. \emph{Inneres} von $A$.
\end{defn}

\begin{defn}
  Ist $(X, \tau)$ ein topologischer Raum und $A \subset X$, dann ist auch $(A, \tau_A)$ ein topologischer Raum mit der \em{Relativtopologie} $\tau_A \coloneqq \{ U \cap A \,|\,U \in \tau \}$.
\end{defn}

\begin{defn}
  Sei $(X, \tau)$ ein topologischer Raum. Eine Teilmenge $A \subset X$ heißt dicht in $X$, falls $\overline{A} = X$.
\end{defn}

\begin{defn}
  Ein topologischer Raum $(X, \tau)$ heißt separabel, falls $X$ eine abzählbare dichte Teilmenge enthält. Eine Teilmenge $A \subset X$ heißt separabel, falls $(A, \tau_A)$ separabel ist.
\end{defn}

\begin{defn}
  Seien $\tau_1, \tau_2$ zwei Topologien auf einer Menge $X$. Dann heißt $\tau_2$ \emph{stärker} (oder feiner) als $\tau_1$ bzw. $\tau_1$ \emph{schwächer} (oder gröber) als $\tau_2$, falls $\tau_1 \subset \tau_2$.
\end{defn}

\begin{defn}
  Seien $d_1$ und $d_2$ Metriken auf einer Menge $X$ und $\tau_1$ und $\tau_2$ die induzierten Topologien. Dann heißt $d_1$ stärker als $d_2$, falls $\tau_1$ stärker ist als $\tau_2$.
\end{defn}

\begin{satz}
  Sind $\| \cdot \|_1$ und $\| \cdot \|_2$ zwei Normen auf dem $\K$-Vektorraum $X$. Dann gilt:
  \begin{itemize}
    \item $\| \cdot \|_2 \text{ ist stärker als } \| \cdot \|_1 \iff \exists\,C > 0 : \forall\,x \in X : \| x \|_1 \leq C \| x \|_2$
    \item $\| \cdot \|_1 \text{ und } \| \cdot \|_2 \text{ sind äquivalent } \iff \exists\,c, C > 0 : \forall\,x \in X : c \| x \|_1 \leq \| x \|_2 \leq C \| x \|_1$
  \end{itemize}
\end{satz}

\begin{defn}
  Die \emph{$p$-Norm} auf dem $\K^n$ ist definiert als
  \begin{align*}
    \| x \|_p &\coloneqq \left( \sum_{i = 1}^n |x_j|^p \right)^{\frac{1}{p}} \text{ für } 1 \leq p < \infty \\
    \| x \|_{\infty} &\coloneqq \| x \|_max \coloneqq \max_{1 \leq i \leq n} |x_i|.
  \end{align*}
\end{defn}

\begin{bem}
  Alle $p$-Normen sind zueinander äquivalent.
\end{bem}

\begin{defn}
  Seien $S \subset X$ eine Menge, $(X, \tau_X)$ und $(Y, \tau_Y)$ Hausdorff-Räume sowie $x_0 \in S$. Eine Funktion $f : S \to Y$ heißt \emph{stetig} in $x_0$, falls gilt:
  \[ \forall\,V \in \tau_Y : f(x_0) \in V \implies \exists\,U \in \tau_X \text{ mit } x_0 \in U \wedge f(U \cap S) \subset V \]
  Ist $X = S$, so heißt $f : X \to Y$ stetige Abbildung, falls $f$ stetig in allen Punkten $x_0 \in X$ ist, d.\,h. $V \in \tau_Y \implies f^{-1}(V) \in \tau_X$.
\end{defn}

\begin{bem}
  In metrischen Räumen ist diese Definition äquivalent zur üblichen Folgendefinition.
\end{bem}

\begin{defn}
  Sei $(X, d)$ ein metrischer Raum. Eine Folge $(x_k)_{k \in \N}$ heißt \emph{Cauchy-Folge}, falls $d(x_k, x_l) \xrightarrow{k, l \to \infty} 0$. Ein Punkt $x \in X$ heißt \emph{Häufungspunkt} der Folge, falls es eine Teilfolge $(x_{k_i})_{i \in \N}$ gibt mit $x_{k_i} - x \xrightarrow{i \to \infty} 0$.
\end{defn}

\begin{defn}
  Ein metrischer Raum $(X, d)$ heißt \emph{vollständig}, falls jede Cauchy-Folge in $X$ einen Häufungspunkt besitzt.
\end{defn}

\begin{defn}
  Ein normierter $\K$-Vektorraum heißt \emph{Banachraum}, falls er vollständig bzgl. der induzierten Metrik ist. Ein Banachraum heißt \emph{Banach-Algebra}, falls er eine Algebra ist mit $\| x \cdot y \|_X \leq \| x \|_x \cdot \| y \|_X$.
\end{defn}

\begin{defn}
  Ein \emph{Hilbertraum} ist ein \emph{Prähilbertraum}, der vollständig bzgl. der vom Skalarprodukt induzierten Norm ist.
\end{defn}

\begin{defn}
  Sei $\K^\N \coloneqq \Set{ (x_n)_{n \in \N} }{ \forall\,i \in \N : x_i \in \K }$ die Menge aller Folgen in $\K$. Mit der Fréchet-Metrik
    \[ \rho(x) \coloneqq \sum_{i = 1}^\infty 2^{-i} \frac{ |x_i| }{ 1 + |x_i| } < 1 \]
  wird der \emph{Folgenraum} $\K^\N$ zu einem Banachraum.
\end{defn}

\begin{satz}
  Sind $(x^k) = (x^k_i)_{i \in \N} \in \K^\N$ und $x = (x_i)_{i \in \N} \in \K^\N$, so gilt
  \[ \rho(x^k - x) \xrightarrow{k \to \infty} 0 \iff \forall\,i \in \N : x_i^k \xrightarrow{k \to \infty} x_i. \]
\end{satz}

\begin{defn}
  Die Norm
  \begin{align*}
    \| x \|_{\ell^p} &\coloneqq \left( \sum_{i=1}^\infty |x_i|^p \right)^{\frac{1}{p}} \in [0, \infty], \text{ für } 1 \leq p < \infty \\
    \| x \|_{\ell^\infty} &\coloneqq \sup_{i \in \N} |x_i| \in [0, \infty]
  \end{align*}
  heißt \emph{$\ell^p$-Norm} auf dem Raum $\ell^p(\K) \coloneqq \Set{ x \in \K^\N }{ \| x \|_{\ell^p} < \infty }$.
\end{defn}

\begin{satz}
  Der Raum $\ell^p(\K)$ ist vollständig, also ein Banachraum.
\end{satz}

\begin{bem}
  Im Fall $p = 2$ wird $\ell^2(\K)$ ein Hilbertraum mit dem Skalarprodukt $\langle x , y \rangle_{\ell^2} \coloneqq \sum_{i = 0}^\infty x_i \overline{y_i}$.
\end{bem}

\begin{defn}[Vervollständigung]
  Sei $(X, d)$ ein metrischer Raum. Betrachte die Menge $X^\N$ aller Folgen in $X$ und definiere
  \[ \tilde{X} \coloneqq \Set{ x \in X^\N }{ x \text{ ist Cauchy-Folge in } X }/\sim \]
  mit der Äquivalenzrelation
  \[ x \sim y \text{ in } \tilde{X} \iff d(x_j, y_j) \xrightarrow{j \to \infty} 0. \]
  Diese Menge wird mit der Metrik
  \[ \tilde{d}(x, y) \coloneqq \lim_{i \to \infty} d(x_i, y_i) \]
  zu einem vollständigen metrischen Raum. Die injektive Abbildung $J : X \to \tilde{X}$, welche $x \in X$ auf die konstante Folge $(x)_{i \in \N}$, ist isometrisch, d.\,h. sie erhält. Wir können also $X$ als einen dichten Unterraum von $\tilde{X}$ auffassen. Man nennt $\tilde{X}$ \emph{Vervollständigung} von~$X$.
\end{defn}

\begin{defn}[\emph{Raum der beschränkten Funktionen}]
  Sei $S$ eine Menge und $Y$ ein Banachraum über $\K$ mit Norm $y \mapsto |y|$. Dann ist
  \[ B(S; Y) \coloneqq \Set{ f : S \to Y }{ f(S) \text { ist eine beschränkte Teilmenge von } Y } \]
  die Menge der beschränkten Funktionen von $B$ nach $Y$. Diese Menge ist ein $\K$-Vektorraum und wird mit der Supremumsnorm $\| f \|_{B(S)} \coloneqq \sup_{x \in S} |f(x)|$ zu einem Banachraum.
\end{defn}

\begin{satz}
  Ist $(X, d)$ ein vollständiger metrischer Raum und $Y \subset X$ abgeschlossen, so ist auch $(Y, d)$ ein vollständiger metrischer Raum.
\end{satz}

\begin{defn}[\emph{Raum stetiger Funktionen auf einem Kompaktum}]
  Sei $S \subset \R^n$ beschränkt und abgeschlossen (d.\,h. kompakt) und $Y$ ein Banachraum über $\K$ mit Norm $y \mapsto |y|$, so ist
  \[ \mathcal{C}^0(S; Y) \coloneqq \mathcal{C}(S; Y) \coloneqq \Set{ f : S \to Y }{ f \text{ ist stetig } } \]
  die Menge der stetigen Funktionen von $S$ nach $Y$. Sie ist ein abgeschlossener Unterraum von $B(S; Y)$ mit der Norm $\| \cdot \|_{\mathcal{C}(S; Y)} = \| \cdot \|_{B(S; Y)}$, also ein Banachraum.
\end{defn}

\begin{bem}
  Für $Y = \K$ ist $\mathcal{C}^0(S; \K) = \mathcal{C}(S)$ eine kommutative Banach-Algebra mit dem Produkt $(f \cdot g)(x) \coloneqq f(x) \cdot g(x)$.
\end{bem}

\begin{defn}
  Sei $S \subset \R^n$ und $(K_n)_{n \in \N}$ eine Folge kompakter Teilmengen des $\R^n$. Dann heißt $(K_n)$ eine \emph{Ausschöpfung} von $S$, falls
  \begin{itemize}
    \item $S = \cup_{n \in \N} K_n$,
    \item $\emptyset \not= K_i \subset K_{i+1} \subset S$ für alle $i \in \N$ und
    \item für alle $x \in S$ gibt es ein $\delta > 0$ und $i \in \N$, sodass $B_\delta(x) \subset K_i$.
  \end{itemize}
\end{defn}

\begin{bem}
  Zu offenen und abgeschlossenen $S \subset \R^n$ existiert eine Ausschöpfung.
\end{bem}

\begin{defn}[\emph{Raum stetiger Funktionen auf Menge mit Ausschöpfung}]
  Es sei $S \subset \R^n$ so, dass eine Ausschöpfung $(K_i)_{i \in \N}$ von $S$ existiert und $Y$ ein Banachraum. Dann bildet die Menge aller stetigen Funktionen
  \[ C^0(S; Y) \coloneqq \Set{ f : S \to Y }{ f \text{ ist stetig auf } S } \]
  einen $K$-Vektorraum und wird mit der Fréchet-Norm
  \[ \varrho(f) \coloneqq \sum_{i \in \N} 2^{-i} \frac{ \| f \|_{C^0(K_i)} }{ 1 + \| f \|_{C^0(K_i)} } \]
  zu einem vollständigen metrischen Raum.
\end{defn}

\begin{bem}
  \begin{itemize}
    \item Die von dieser Metrik erzeugte Topologie ist unabhängig von der Wahl der Ausschöpfung.
    \item Ist $S \subset \R^n$ kompakt, so stimmt die Topologie mit der von $\| \cdot \|_{B(s)}$ überein.
  \end{itemize}
\end{bem}

\begin{defn}
  Sei $S \subset \R^n$ und $Y$ ein Banachraum. Für $f : S \to Y$ heißt
  \[ \supp f \coloneqq \Set{ x \in S }{ f(x) \not= 0 } \]
  \emph{Träger} (engl. support) von $f$.
\end{defn}

\begin{defn}
  Sei $S \subset \R^n$ und $Y$ ein Banachraum. Dann ist
  \[ \mathcal{C}_0^0(S; Y) \coloneqq \Set{ f \in \mathcal{C}^0(S; Y) }{ \supp f \text{ ist kompakt in } S } \]
  die Menge der stetigen Funktionen mit kompaktem Träger von $S$ nach $Y$.
\end{defn}

\begin{defn}[\emph{Raum differenzierbarer Funktionen}]
  Sei $\Omega \subset \R^n$ offen und beschränkt und $m \in \N$. Dann ist die Menge der differenzierbaren Funktionen von $\Omega$ nach $Y$
  \begin{align*}
    \mathcal{C}^m(\overline\Omega, Y) \coloneqq \{ f : \Omega \to Y \,|\, & f \text{ ist $m$-mal stetig differenzierbar in $\Omega$ } \\
    & \text{ und für $k \leq m$ und } s_1, ..., s_k \in \{ 1, ..., n \} \\
    & \text{ ist $\partial_{s_1} ... \partial_{s_k} f$ auf $\overline\Omega$ stetig fortsetzbar } \}
  \end{align*}
  ein Vektorraum und mit
  \[ \| f \|_{\mathcal{C}^m(\overline\Omega)} = \sum_{|s| \leq m} \| \partial^s \|_{\mathcal{C}^0(\overline\Omega)} \]
  ein Banachraum
\end{defn}

\begin{bem}
  In obiger Norm wird die Summe über alle $k$-fache partielle Ableitungen mit $k \leq m$ gebildet.
\end{bem}

% TODO: Hölder-Räume
% TODO: Unendlich differenzierbare Funktionen
% TODO: Lebesgue-Räume
% TODO: Sobolev-Räume

\end{document}
