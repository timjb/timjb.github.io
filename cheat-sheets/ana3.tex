\documentclass[a4paper,10pt,landscape]{article}
\usepackage{multicol}
\usepackage{calc}
\usepackage{ifthen}
\usepackage[landscape]{geometry}
\usepackage{amsmath,amsthm,amsfonts,amssymb}
\usepackage{color,graphicx,overpic}
\usepackage{hyperref}
\usepackage[utf8]{inputenc}
\usepackage[ngerman]{babel}
\usepackage{enumitem}
\usepackage{mathtools}
\usepackage{pifont}

\setitemize[0]{leftmargin=10pt,itemindent=0pt,itemsep=0pt}
\setenumerate[0]{leftmargin=10pt,itemindent=0pt,itemsep=0pt}

\newcommand{\cmark}{\ding{51}}
\newcommand{\xmark}{\ding{55}}
\newcommand{\R}{\mathbb{R}}
\newcommand{\N}{\mathbb{N}}
\newcommand{\Z}{\mathbb{Z}}
\newcommand{\C}{\mathbb{C}}
\newcommand{\Q}{\mathbb{Q}}
\newcommand{\PS}{\mathcal{P}} % Powerset
\newcommand{\PSO}{\PS(\Omega)} % Powerset
\newcommand{\Alg}{\mathfrak{A}}
\newcommand{\Ring}{\mathfrak{R}}

% Differentiator
\renewcommand{\d}{\mathrm{d}}

\pdfinfo{
  /Title (ana3.pdf)
  /Creator (TeX)
  /Producer (pdfTeX 1.40.0)
  /Author (Tim Baumann)
  /Subject (Analysis 3 Zusammenfassung)
  /Keywords (analysis,overview)}

% This sets page margins to .5 inch if using letter paper, and to 1cm
% if using A4 paper. (This probably isn't strictly necessary.)
% If using another size paper, use default 1cm margins.
\ifthenelse{\lengthtest { \paperwidth = 11in}}
    { \geometry{top=.5in,left=.5in,right=.5in,bottom=.5in} }
    {\ifthenelse{ \lengthtest{ \paperwidth = 297mm}}
        {\geometry{top=1cm,left=1cm,right=1cm,bottom=1cm} }
        {\geometry{top=1cm,left=1cm,right=1cm,bottom=1cm} }
    }

\theoremstyle{definition}

\newtheorem*{nota}{Notation}
\newtheorem*{defn}{Definition}
\newtheorem*{prob}{Problem}
\newtheorem*{bsp}{Beispiel}
\newtheorem*{satz}{Satz}
\newtheorem*{kor}{Korollar}
\newtheorem*{acht}{Achtung}
\newtheorem*{strat}{Strategie}

\theoremstyle{remark}
\newtheorem*{bem}{Bemerkung}

% Römische Ziffern
\makeatletter
\newcommand*{\rom}[1]{\expandafter\@slowromancap\romannumeral #1@}
\makeatother

% Ober- und Unterintegral, siehe
% http://tex.stackexchange.com/questions/44237/lower-and-upper-riemann-integrals
\def\upint{\mathchoice%
    {\mkern13mu\overline{\vphantom{\intop}\mkern7mu}\mkern-20mu}%
    {\mkern7mu\overline{\vphantom{\intop}\mkern7mu}\mkern-14mu}%
    {\mkern7mu\overline{\vphantom{\intop}\mkern7mu}\mkern-14mu}%
    {\mkern7mu\overline{\vphantom{\intop}\mkern7mu}\mkern-14mu}%
  \int}
\def\lowint{\mkern3mu\underline{\vphantom{\intop}\mkern7mu}\mkern-10mu\int}

% Färbe \emph{}
\definecolor{Emph}{rgb}{0.2,0.2,0.8}  %softer red for display
\renewcommand{\emph}[1]{\textcolor{Emph}{\bf{#1}}}

% Display style überall!
\everymath{\displaystyle}

% Turn off header and footer
\pagestyle{empty}

% Redefine section commands to use less space
\makeatletter
\renewcommand{\section}{\@startsection{section}{1}{0mm}%
                                {-1ex plus -.5ex minus 2ex}%
                                {2.5ex plus .2ex}%x
                                {\normalfont\large\bfseries}}
\renewcommand{\subsection}{\@startsection{subsection}{2}{0mm}%
                                {-1explus -.5ex minus -.2ex}%
                                {0.5ex plus .2ex}%
                                {\normalfont\normalsize\bfseries}}
\renewcommand{\subsubsection}{\@startsection{subsubsection}{3}{0mm}%
                                {-1ex plus -.5ex minus -.2ex}%
                                {1ex plus .2ex}%
                                {\normalfont\small\bfseries}}
\makeatother

% Don't print section numbers
\setcounter{secnumdepth}{0}

\DeclarePairedDelimiterX\Set[2]{\lbrace}{\rbrace}%
 { #1 \,\delimsize|\, #2 }

\setlength{\parindent}{0pt}
\setlength{\parskip}{0pt plus 10ex}

% -----------------------------------------------------------------------

\begin{document}
\raggedright
\footnotesize
\begin{multicols}{3}

% multicol parameters
% These lengths are set only within the two main columns
%\setlength{\columnseprule}{0.25pt}
\setlength{\premulticols}{1pt}
\setlength{\postmulticols}{1pt}
\setlength{\multicolsep}{1pt}
\setlength{\columnsep}{2pt}

\begin{center}
  \Large{\underline{Zusammenfassung Analysis \rom{3}}} \\
\end{center}

\section{Maßtheorie}

\begin{prob}[Schwaches Maßproblem]
  Gesucht: Abbildung $\mu : \PS(\R^n) \to [\R, \infty]$ mit folgenden Eigenschaften:
  \begin{itemize}
    \item Normierung: $\mu([0, 1]^n) = 1$
    \item Endliche Additivität: Sind $A, B \subset \R^n$ disjunkt, so gilt $\mu(A \cup B) = \mu(A) + \mu(B)$
    \item Bewegungsinvarianz: Für eine euklidische Bewegung $f : \R^n \to \R^n$ und $A \subset \R^n$ gilt $\mu(f(A)) = \mu(A)$.
  \end{itemize}
\end{prob}

\begin{satz}[Hausdorff]
  Das schwache Maßproblem ist für $n \geq 3$ nicht lösbar.
\end{satz}

\begin{satz}[Banach]
  Das schwache Maßproblem ist für $n = 1, 2$ lösbar, aber nicht eindeutig lösbar.
\end{satz}

\begin{prob}[Starkes Maßproblem]
  Gesucht ist eine Abbildung $\mu : \PS(\R^n) \to [0, \infty]$ wie im schwachen Maßproblem, die anstelle der endlichen Additivität die Eigenschaft der $\sigma$-Additivität besitzt:
  \begin{itemize}
    \item Für eine Folge $(A_n)_{n \in \N}$ paarweise disjunkter Teilmengen des $\R^n$ ist
      \[ \mu\left(\bigcup_{n \in \N} A_n\right) = \sum_{n=0}^\infty \mu(A_n) \]
  \end{itemize}
\end{prob}

\begin{satz}
  Das starke Maßproblem besitzt keine Lösung.
\end{satz}

\begin{nota}
  Sei im Folgenden $\Omega$ eine Menge.
\end{nota}

\begin{defn}
  Eine Teilmenge $\Ring \subset \PSO$ heißt \emph{Ring}, wenn für $A, B \in \Ring$ gilt:
  \begin{itemize}
    \item $\emptyset \in \Ring$
    \item Abgeschlossenheit unter Differenzbildung: $A \setminus B \in \Ring$
    \item Abgeschlossenheit unter endlichen Vereinigungen: $A \cup B \in \Ring$
  \end{itemize}
\end{defn}

\begin{defn}
  Eine Teilmenge $\Alg \subset \PSO$ heißt \emph{Algebra}, wenn für $A, B \in \Alg$ gilt:
  \begin{itemize}
    \item $\emptyset \in \Alg$
    \item Abgeschlossenheit unter Komplementbildung: $A^c = \Omega \setminus A \in \Alg$
    \item Abgeschlossenheit unter endlichen Vereinigungen: $A \cup B \in \Alg$
  \end{itemize}
\end{defn}

\begin{defn}
  Eine Algebra $\Alg \subset \PSO)$ heißt \emph{$\sigma$-Algebra}, wenn $\Alg$ unter abzählbaren Vereinigungen abgeschlossen ist, d.\,h. für jede Folge $(A_n)_{n \in \N}$ in $\Alg$ gilt

  \[ \bigcup_{n \in \N} A_n \in \Alg \]
\end{defn}

\begin{bem}
  \begin{itemize}
    \item Jede Algebra ist auch ein Ring.
    \item Ein Ring $\Ring \subset \PSO$ ist auch unter endlichen Schnitten abgeschlossen, da $A \cap B = A \setminus (B \setminus A) \in \Ring$
    \item Ein Ring $\Ring \subset \PSO$ ist genau dann eine Algebra, wenn $\Omega \in \Ring$
    \item Eine $\sigma$-Algebra $\Alg \subset \PSO$ ist auch unter abzählbaren Schnitten abgeschlossen: Sei $(A_n)_{n \in \N}$ eine Folge in $\Alg$, dann gilt
      \[ \bigcap_{n \in \N} A_n = \left( \bigcup_{n \in \N} (A_n)^c \right)^c \in \Alg \]
  \end{itemize}
\end{bem}

\begin{nota}
  Sei im Folgenden $\Ring \subset \PSO$ ein Ring.
\end{nota}

% Lemma 1.10 (Urbildalgebra)

\begin{satz}
  Sei $(A_i)_{i \in I}$ eine Familie von Ringen / Algebren / $\sigma$-Algebren über $\Omega$. Dann ist auch $\cap_{i \in I} A_i$ ein Ring / eine Algebra / eine $\sigma$-Algebra über $\Omega$.
\end{satz}

\begin{defn}
  Sei $E \subset \PSO$. Setze
  \begin{align*}
    \mathcal{R}(E) &:= \Set{ \Ring \subset \PSO }{ E \subset \Ring, \Ring \text{ Ring} } \text{ und} \\
    \mathcal{A}(E) &:= \Set{ \Alg \subset \PSO }{ E \subset \Alg, \Alg \text{ $\sigma$-Algebra} }.
  \end{align*}
  Dann heißen
  \[
    \Ring(E) :=\!\bigcap_{\Ring \in \mathcal{R}(E)}\!\Ring, \qquad
    \Alg(E)  :=\!\bigcap_{\Alg  \in \mathcal{A}(E)}\!\Alg
  \]
  von $E$ \emph{erzeugter Ring} bzw. von $E$ \emph{erzeugte $\sigma$-Algebra}.
\end{defn}

\begin{defn}
  Ist $(\Omega, \mathcal{O})$ ein topologischer Raum, dann heißt $\mathfrak{B} = \mathfrak{B}(\Omega, \mathcal{O}) := \Alg(\mathcal{O})$ \emph{Borelsche $\sigma$-Algebra} von $(\Omega, \mathcal{O})$.
\end{defn}

\begin{bem}
  Die Borelsche $\sigma$-Algebra $\mathfrak{B}(\R)$ wird auch erzeugt von $\Set{I \subset \R }{ I \text{ Intervall } }$. Dabei spielt es keine Rolle, ob man nur geschlossene, nur offene, beliebig halboffene Intervalle oder gar nur Intervalle mit Endpunkten in $\Q$ zulässt.
\end{bem}

\begin{defn}
  Eine Funktion $\mu : \Ring \to [0, \infty]$ heißt \emph{Inhalt} auf $\Ring$, falls
  \begin{itemize}
    \item $\mu(\emptyset) = 0$ und
    \item $\mu(A \sqcup B) = \mu(A) + \mu(B)$ für disjunkte $A, B \in \Ring$.
  \end{itemize}
\end{defn}

\begin{defn}
  Ein Inhalt $\mu : \Ring \to [0, \infty]$ heißt \emph{Prämaß} auf $\Ring$, wenn $\mu$ $\sigma$-additiv ist, d.\,h. wenn für jede Folge $(A_n)_{n \in \N}$ paarweise disjunkter Elemente von $\Ring$ mit $\sqcup_{n \in \N} A_n \in \Ring$ gilt:
  \[ \mu\left(\bigsqcup_{n \in \N} A_n\right) = \sum_{n=0}^\infty \mu(A_n) \]
\end{defn}

\begin{defn}
  Ein \emph{Maß} ist ein Prämaß auf einer $\sigma$-Algebra.
\end{defn}

\begin{satz}
  Für einen Inhalt $\mu$ auf $\Ring$ gilt für alle $A, B \in \Ring$:
  \begin{itemize}
    \item $\mu(A \cup B) + \mu(A \cap B) = \mu(A) + \mu(B)$
    \item Monotonie: $A \subset B \implies \mu(A) \leq \mu(B)$
    \item Aus $A \subset B$ und $\mu(B) < \infty$ folgt $\mu(B \setminus A) = \mu(B) - \mu(A)$
    \item Subadditivität: Für $A_1, ..., A_n \in \Ring$ ist $\mu\left(\bigcup_{i = 1}^n A_i \right) \leq \sum_{i = 1}^n \mu(A_i)$
    \item Ist $(A_n)_{n \in \N}$ eine Folge disjunkter Elemente aus $\Ring$, sodass $\bigsqcup_{n \in \N} A_n \in \Ring$, so gilt $\mu(\bigsqcup_{n \in \N} A_n) \geq \sum_{n=0}^\infty \mu(A_n)$.
  \end{itemize}
\end{satz}

\begin{defn}
  Ein Inhalt / Maß auf einem Ring~$\Ring$ / einer $\sigma$-Algebra~$\Alg$ heißt \emph{endlich}, falls $\mu(A) < \infty$ für alle $A \in \Ring$ bzw. $A \in \Alg$.
\end{defn}

\begin{satz}
  Ein Maß auf einer $\sigma$-Algebra~$\Alg$ ist $\sigma$-subadditiv, d.\,h. für alle Folgen $(A_n)_{n \in \N}$ in $\Alg$ gilt
  \[ \mu(\bigcup_{n \in \N} A_n) \leq \sum_{n=0}^\infty \mu(A_n). \]
\end{satz}

\begin{defn}
  Sei $A \subset \Omega$. Dann heißt die Abbildung
  \[ 1_A : \Omega \to \R, \quad \omega \mapsto \begin{cases} 1, & \text{ falls } \omega \in A \\ 0, & \text{ falls } \omega \not\in A \end{cases} \]
  \emph{Indikatorfunktion} oder \emph{charakteristische Funktion} von $A$.
\end{defn}

\begin{defn}
  Wir sagen eine Folge $(A_n)_{n \in \N}$ \emph{konvergiert} gegen $A \subset \Omega$, notiert $\lim_{n \to \infty} A_n = A$, wenn $(1_{A_n})_{n \in \N}$ punktweise gegen $1_A$ konvergiert.
\end{defn}

\begin{defn}
  Für eine Folge $(A_n)_{n \in \N}$ in $\PS(\Omega)$ heißen
  \begin{align*}
    \limsup_{n \to \infty} A_n :=&~\Set{ \omega \in \Omega }{ \omega \text{ liegt in undendlich vielen } A_n } \\
    =&~\bigcap_{n = 0}^\infty \bigcup_{k = n}^\infty A_n \\
    \liminf_{n \to \infty} A_n :=&~\Set{ \omega \in \Omega }{ \omega \text{ liegt in allen bis auf endlich vielen } A_n } \\
    =&~\bigcup_{n = 0}^\infty \bigcap_{k = n}^\infty A_n
  \end{align*}
  \emph{Limes Superior} bzw. \emph{Limes Inferior} der Folge $A_n$.
\end{defn}

% TODO: Lemmas zum Limes Inferior, Limes Superior

\begin{satz}
  Es gilt $\lim_{n \to \infty} A_n = A \iff \liminf_{n \to \infty} A_n = \limsup_{n \to \infty} A_n = A$.
\end{satz}

\begin{defn}
  Eine Folge $(A_n)_{n \in \N}$ in $\PSO$ heißt
  \begin{itemize}
    \item \emph{monoton wachsend}, wenn für alle $n \in \N$ gilt $A_n \subset A_{n+1}$,
    \item \emph{monoton fallend}, wenn für alle $n \in \N$ gilt $A_n \supset A_{n+1}$.
  \end{itemize}
\end{defn}

\begin{satz}
  Sei $(A_n)_{n \in \N}$ eine Folge in $\PSO$.
  \begin{itemize}
    \item Ist $(A_n)$ monoton wachsend, so gilt $\lim_{n \to \infty} A_n = \bigcup_{n \in \N} A_n$.
    \item Ist $(A_n)$ monoton fallend, so gilt $\lim_{n \to \infty} A_n = \bigcap_{n \in \N} A_n$.
  \end{itemize}
\end{satz}

\begin{satz}
  Sei $\mu$ ein Inhalt auf $\Ring \subset \PSO$. Wir betrachten folgende Aussagen:

  \begin{enumerate}[label=(\roman*),leftmargin=2em]
    \item $\mu$ ist ein Prämaß auf $\Ring$.
    \item Stetigkeit von unten: Für jede monoton wachsende Folge $(A_n)_{n \in \N}$ in $\Ring$ mit $A := \lim_{n \to \infty} A_n = \bigcup_{n = 0}^\infty A_n \in \Ring$ gilt $\lim_{n \to \infty} \mu(A_n) = \mu(A)$.
    \item Stetigkeit von oben: Für jede monoton fallende Folge $(A_n)_{n \in \N}$ in $\Ring$ mit $\mu(A_0) < \infty$ und $A := \lim_{n \to \infty} A_n = \bigcap_{n = 0}^\infty A_n \in \Ring$ gilt $\lim_{n \to \infty} \mu(A_n) = \mu(A)$.
    \item Für jede monoton fallende Folge $(A_n)_{n \in \N}$ in $\Ring$ mit $\mu(A_0) < \infty$ und $\lim_{n \to \infty} A_n = \bigcap_{n = 0}^\infty A_n = \emptyset$ gilt $\lim_{n \to \infty} \mu(A_n) = 0$.
  \end{enumerate}

  Dann gilt $(i) \iff (ii) \implies (iii) \iff (iv)$.\\
  Falls $\mu$ endlich, gilt auch $(iii) \implies (ii)$.
\end{satz}

\begin{satz}
  Sei $\mu$ ein Maß auf einer $\sigma$-Algebra $\Alg \subset \PSO$. Dann gilt:
  \begin{itemize}
    \item Für eine Folge $(A_n)_{n \in \N}$ in $\Alg$ gilt $\mu\left(\liminf_{n \to \infty} A_n\right) \leq \liminf_{n \to \infty}(\mu(A_n))$.
    \item Sei $(A_n)_{n \in \N}$ eine Folge in $\Alg$, sodass es ein $N \in \N$ gibt mit $\mu\left(\bigcup_{n = N}^\infty A_n \right) < \infty$, dann gilt $\mu\left(\limsup_{n \to \infty} A_n \right) \geq \limsup_{n \to \infty} \mu(A_n)$.
    \item Sei $\mu$ endlich und $(A_n)_{n \in \N}$ eine Folge in $\Alg$, dann gilt
    \[ \mu\left(\liminf_{n \to \infty} A_n\right) \leq \liminf_{n \to \infty} \mu(A_n) \leq \limsup_{n \to \infty} \mu(A_n) \leq \mu\left(\limsup_{n \to \infty} A_n\right). \]
    \item Sei $\mu$ endlich und $(A_n)_{n \in \N}$ eine gegen $A$ konvergente Folge in $\Alg$, dann gilt $A \in \Alg$ und $\mu(A) = \lim_{n \to \infty} \mu(A_n)$.
  \end{itemize}
\end{satz}

\begin{defn}
  Ein Inhalt auf einem Ring $\Ring \subset \PSO$ heißt \emph{$\sigma$-endlich}, wenn gilt: Es gibt eine Folge $(S_n)_{n \in \N}$ in $\Ring$, sodass
  \begin{itemize}
    \item $\Omega = \bigcup_{n \in \N} S_n$ und
    \item $\mu(S_n) < \infty$ für alle $n \in \N$
  \end{itemize}
\end{defn}

\begin{defn}
  Eine Funktion $f : \Omega \to \overline{\R} = \R \cup \{ \pm \infty \}$ wird \emph{numerische Funktion} genannt.
\end{defn}

\begin{defn}
  Eine numerische Funktion $\mu^* : \PSO \to \overline{\R}$ heißt \emph{äußeres Maß} auf $\Omega$, wenn gilt:
  \begin{itemize}
    \item $\mu^*(\emptyset) = 0$
    \item Monotonie: $A \subset B \implies \mu^*(A) \leq \mu^*(B)$
    \item $\sigma$-Subadditivität: Ist $(A_n)_{n \in \N}$ eine Folge von Teilmengen von $\Omega$, dann gilt $\mu^*\left(\bigcup_{n \in \N} A_n \right) \leq \sum_{n = 0}^\infty \mu^*(A_n)$
  \end{itemize}
\end{defn}

\begin{bem}
  Wegen $\mu^*(\emptyset) = 0$ und der Monotonie nimmt ein äußeres Maß nur Werte in $[0, \infty]$ an.
\end{bem}

\begin{defn}
  Eine Teilmenge $A \subset \Omega$ heißt \emph{$\mu^*$-messbar}, falls für alle $Q \subset \Omega$ gilt
  \[ \mu^*(Q) = \mu^*(Q \cap A) + \mu^*(Q \setminus A). \]
\end{defn}

\begin{satz}[Carathéodory]
  Sei $\mu^* : \PSO \to [0, \infty]$ ein äußeres Maß, dann gilt
  \begin{itemize}
    \item Die Menge $\Alg^* := \Set{ A \subset \Omega }{ A \text{ ist $\mu^*$-messbar } }$ ist eine $\sigma$-Algebra.
    \item $\mu^*|_{\Alg^*}$ ist ein Maß auf $\Alg^*$.
  \end{itemize}
\end{satz}

\begin{satz}[\emph{Fortsetzungssatz}]
  Sei $\mu$ ein Prämaß auf einem Ring $\Ring$, dann gibt es ein Maß $\tilde{\mu}$ auf der von $\Ring$ erzeugten $\sigma$-Algebra $\Alg(\Ring)$ mit $\tilde{\mu}|_\Ring = \mu$. Falls $\mu$ $\sigma$-endlich, so ist $\tilde{\mu}$ eindeutig bestimmt.
\end{satz}

\begin{bem}
  Im Beweis wird ein äußeres Maß auf $\Omega$ wie folgt definiert:

  \[ \mu^*(Q) := \inf \left\{ \sum_{i = 0}^\infty \mu(A_n) \,\middle|\, (A_n)_{n \in \N} \in \mathfrak{U}(Q) \right\}, \]

  wobei $\inf \emptyset := \infty$ und

  \[ \mathfrak{U}(Q) := \left\{ (A_n)_{n \in \N} \middle| Q \subset \bigcup_{n = 0}^\infty A_n \text{ und } A_n \text{ Folge in } \Ring \right\}. \]

  Das Prämaß $\mu^*$ eingeschränkt auf $\Alg^* \supset \Alg(\Ring)$ ist ein Maß.
\end{bem}


\subsection{Das Lebesgue-Borel-Maß}

\begin{nota}
  Für zwei Elemente $a = (a_1, ..., a_n)$ und $b = (b_1, ..., b_n)$ schreibe
  \begin{itemize}
    \item $a \lhd b$, falls $a_j < b_j$ für alle $j = 1, ..., n$.
    \item $a \unlhd b$, falls $a_j \leq b_j$ für alle $j = 1, ..., n$.
  \end{itemize}
\end{nota}

\begin{defn}
  Für $a, b \in \R^n$ heißen
  \begin{align*}
    ]a, b[\,:=& \Set{ x \in \R^n }{ a \lhd x \lhd b }\\
    \mu(]a, b[) :=& \prod_{j = 1}^{n} (b_j - a_j)
  \end{align*}
  \emph{Elementarquader} und \emph{Elementarinhalt}. Sei im Folgenden $\mathcal{E}$ die Menge aller Elementarquader.
\end{defn}

\begin{satz}
  Für alle $A \in \Ring(\mathcal{E})$ gibt es paarweise disjunkte Elementarquader $Q_1, ..., Q_p \in \mathcal{E}$ sodass $A = \bigsqcup_{i = 1}^p Q_i$.
\end{satz}

\begin{defn}
  Für $A \in \Ring(\mathcal{E})$ setze $\mu(A) := \sum_{i = 1}^p \mu(Q_i)$, wenn $A = \bigsqcup_{i = 1}^p Q_i$ für paarweise disjunkte $Q_1, ..., Q_p$.
\end{defn}

\begin{satz}
  $\mu$ definiert ein Prämaß auf $\Ring(\mathcal{E})$, genannt das \emph{Lebesgue-Borel-Prämaß} auf $\R^n$.
\end{satz}

\begin{defn}
  Die eindeutige (da $\mu$ $\sigma$-endlich) Fortsetzung $\tilde{\mu}$ von $\mu$ auf $\Alg(\mathcal{E}) = \mathfrak{B}(\R^n)$ wird \emph{Lebesgue-Borel-Maß} genannt.
\end{defn}

\begin{bem}
  Das Lebesgue-Borel-Maß ist das einzige Maß auf $\mathfrak{B}(\R^n)$, welches jedem Elementarquader seinen Elementarinhalt zuordnet.
\end{bem}

\begin{defn}
  Sei $\mu$ ein Maß auf einer $\sigma$-Algebra $\Alg \subset \PSO$. Eine Menge $N \subset \Omega$ heißt \emph{Nullmenge}, wenn es eine Menge $A \in \Alg$ gibt mit $N \subset A$ und $\mu(A) = 0$. Die Menge aller Nullmengen wird mit $\mathfrak{N}_\mu$ bezeichnet.
\end{defn}

\begin{defn}
  Sei $\mu$ ein Maß auf einer $\sigma$-Algebra $\Alg$. Setze
  \[ \tilde{\Alg}_\mu := \Set{ A \cup N }{ A \in \Alg, N \in \mathfrak{N}_\mu }. \]
  Dann gilt:
  \begin{itemize}
    \item $\tilde{\Alg}_\mu = \Alg(\mathfrak{N}_\mu \cup \Alg)$, ist also eine $\sigma$-Algebra.
    \item Die Funktion $\mu : \tilde{\Alg}_\mu \to [0, \infty]$ definiert durch $\tilde\mu(\tilde{A}) := \mu(A)$, wenn $\tilde{\Alg}_\mu \ni A \cup N$ mit $A \in \Alg$ und $N \in \mathfrak{N}$, ist ein Maß.
  \end{itemize}
\end{defn}

\begin{defn}[Fortsetzung auf Nullmengen]
  Sei $\mu$ das Lebesgue-Borel-Maß auf $\mathfrak{B}(\R^n)$. Dann heißt die von $\mathfrak{B}(\R^n)$ und den entsprechenden Nullmengen erzeugte $\sigma$-Algebra $\tilde{\Alg}_\mu$ \emph{Lebesguesche $\sigma$-Algebra}, notiert $\mathfrak{L}(\R^n)$, und das fortgesesetzte Maß \emph{Lebesgue-Maß}.
\end{defn}

\begin{defn}
  Sei $\Omega$ eine Menge und $\Alg \subset \PSO$ eine $\sigma$-Algebra auf $\Omega$, sowie ggf. $\mu$ ein Maß auf $\Alg$. Dann heißt
  \begin{itemize}
    \item das Tupel $(\Omega, \Alg)$ \emph{messbarer Raum},
    \item das Tripel $(\Omega, \Alg, \mu)$ \emph{Maßraum}.
  \end{itemize}
\end{defn}

\begin{defn}
  Seien $(\Omega, \Alg)$ und $(\Omega', \Alg')$ zwei messbare Räume. Eine Abbildung $f : \Omega \to \Omega'$ heißt \emph{messbar} oder genauer $(\Alg, \Alg')$-messbar, wenn für alle $A' \in \Omega'$ gilt $f^{-1}(A') \in \Omega$ oder, kürzer, $f^{-1}(\Alg') \subset \Alg$.
\end{defn}

\end{multicols}
\end{document}
