\documentclass{cheat-sheet}

\pdfinfo{
  /Title (Zusammenfassung Differentialgeometrie)
  /Author (Tim Baumann)
}

\newcommand{\Intabdt}[1]{\Int{a}{b}{#1}{t}}
\newcommand{\Bil}{\mathrm{Bil}}
\newcommand{\SymBil}{\mathrm{SymBil}}
\newcommand{\I}{\mathrm{I}}
\newcommand{\II}{\mathrm{I\!I}}
\newcommand{\spur}{\mathrm{spur}}
\newcommand{\Graph}{\mathrm{Graph}}

% Offene und abgeschlossene Teilmengen
% http://tex.stackexchange.com/questions/22371/subseteq-circ-as-a-single-symbol-open-subset
\newcommand\opn{\mathrel{\ooalign{$\subset$\cr
  \hidewidth\raise.1ex\hbox{$\circ\mkern.5mu$}\cr}}}
\newcommand\cls{\mathrel{\ooalign{$\subset$\cr
  \hidewidth\raise.1ex\hbox{$\bullet\mkern.5mu$}\cr}}}

\begin{document}

\maketitle{Zusammenfassung Geometrie}


\section{Geometrie von Kurven}

% Erinnerung: Skalarprodukt, Isometrie, Winkel, Orthogonalität

\begin{nota}
  Sei im Folgenden $I$ ein Intervall, d.\,h. eine zusammenhängende Teilmenge von $\R$.
\end{nota}

\begin{defn}
  Eine Abbildung $c : I \to \R^n$ heißt \emph{reguläre Kurve}, wenn $c$ beliebig oft differenzierbar ist und $c'(t) \not= 0$ für alle $t \in I$ gilt.

  Der affine Unterraum $\tau_{c,t} \coloneqq c(t) + \R(c'(t))$ heißt \emph{Tangente} an $c$ im Punkt $c(t)$ bzw. Tangente an $c$ zum Zeitpunkt $t$.
\end{defn}

% Lemma 1.2: Tangenten ändern sich unter Parameterwechseln nicht

\begin{defn}
  Die \emph{Bogenlänge} (BL) einer regulären Kurve $c : [a, b] \to \R^n$ ist
  \[ L(c) \coloneqq \Intabdt{\|c'(t)\|}. \]
\end{defn}

\begin{satz}
  Die Bogenlänge ist invariant unter Umparametrisierung, d.\,h. sei $c : [a_2, b_2] \to \R^n$ eine reguläre Kurve und $\phi : [a_1, b_1] \to [a_2, b_2]$ ein Diffeomorphismus, dann gilt $L(c) = L(c \circ \phi)$.
\end{satz}

\begin{defn}
  Eine reguläre Kurve $c : I \to \R^n$ heißt \emph{nach Bogenlänge parametrisiert}, wenn $\| c'(t) \| = 1$ für alle $t \in I$.
\end{defn}

\begin{satz}
  Jede reguläre Kurve $c : I \to \R$ lässt sich nach BL parametrisieren, d.\,h. es existiert ein Intervall $J$ und ein Diffeomorphismus $\phi : J \to I$, welcher sogar orientierungserhaltend ist, sodass $\tilde{c} \coloneqq c \circ \phi$ nach BL parametrisiert ist.
\end{satz}

\begin{defn}
  Zwei Vektoren $a, b \in \R^n$ heißen gleichgerichtet, falls $a = \lambda b$ für ein $\lambda \geq 0$.
\end{defn}

\begin{satz}
  Sei $v : [a, b] \to \R^n$ stetig, dann gilt
  \[ \| \Intabdt{v(t)} \| \leq \Intabdt{\|v(t)\|}, \]
  wobei Gleichheit genau dann gilt, falls alle $v(t)$ gleichgerichtet sind.
\end{satz}

\begin{satz}
  Sei $c : [a, b] \to \R^n$ eine reguläre Kurve und $x \coloneqq c(a), y \coloneqq c(b)$. Dann gilt $L(c) \geq d(x, y)$. Wenn $L(c) = d(x, y)$, dann gibt es einen Diffeomorphismus $\phi : [a, b] \to [0, 1]$, sodass
  \[ c = c_{xy} \circ \phi, \]
  wobei $c_{xy} : [0, 1] \to \R^n,\,t \mapsto x + t (y - x)$.
\end{satz}

% TODO: Definition Feinheit

\begin{defn}
  Sei $c : [a, b] \to \R^n$ eine stetige Kurve und $a = t_0 < t_1 < ... < t_k = b$ eine Zerteilung von $[a, b]$. Dann ist die Länge des \emph{Polygonzugs} durch die Punkte $c(t_i)$ gegeben durch
  \[ \hat{L}_c(t_0, ..., t_k) = \sum_{j=1}^k \| c(t_j) - c(t_{j-1}) \|. \]
\end{defn}

\begin{defn}
  Eine stetige Kurve $c$ heißt \emph{rektifizierbar} von Länge $\hat{L}_c$, wenn gilt: Für alle $\epsilon > 0$ gibt es ein $\delta > 0$, sodass für alle Unterteilungen $a = t_0 < t1 < ... < t_k = b$ der Feinheit mindestens $\delta$ gilt:
  \[ \| \hat{L}_c - \hat{L}_c(t_0, t_1, ..., t_k) \| < \epsilon. \]
\end{defn}

\begin{defn}
  Sei $c : I \to \R^n$ regulär und nach BL parametrisiert. Dann heißt der Vektor $c''(t)$ \emph{Krümmungsvektor} von $c$ in $t \in I$ und die Abbildung $\kappa : I \to \R, \quad t \mapsto \| c''(t) \|$ heißt \emph{Krümmung} der nach BL parametrisierten Kurve.
\end{defn}

\begin{defn}
  Eine Kurve $c : I \to \R^2$ wird \emph{ebene Kurve} genannt.
\end{defn}

\begin{defn}
  Sei $c$ eine reguläre, nach BL parametrisierte, ebene Kurve. Dann heißt
  \[ n = n_c : I \to \R^2, \quad t \mapsto J \cdot c'(t) \text{ mit } J \coloneqq \begin{pmatrix} 0 & -1 \\ 1 & 0 \end{pmatrix}  \]
  das \emph{Normalenfeld} von $c$.
\end{defn}

\begin{bem}
  Für alle $t \in I$ bildet $(c'(t), n_c(t))$ eine positiv orientierte Orthonormalbasis des $\R^2$.
  Es gilt außerdem $c''(t) \perp c'(t)$, also $c''(t) = \kappa(t) \cdot n_c(t)$, d.\,h. die Krümmung ist im $\R^2$ vorzeichenbehaftet.
\end{bem}

\begin{satz}[\emph{Frenet-Gleichungen} ebener Kurven]
  Sei $c : I \to \R^2$ regulär, nach BL parametrisiert und $v = c'$, dann gilt
  \[ c'' = \kappa \cdot n \quad \text{ und } \quad n' = -\kappa \cdot v. \]
\end{satz}

% Matrixschreibweise?
% Beispiel: Kreis

\begin{bsp}
  Die nach BL parametrisierte gegen den UZS durchlaufene Kreislinie mit Mittelpunkt $m \in \R^2$ und Radius $r > 0$
  \[ c : \R \to \R^2, \quad t \mapsto m + r \begin{pmatrix} \cos(t/r) \\ \sin(t/r) \end{pmatrix} \]
  hat konstante Krümmung $\kappa(t) = \tfrac{1}{r}$.
\end{bsp}

\begin{satz}
  Sei $c : I \to \R^2$ glatte, nach BL parametrisiert mit konstanter Krümmung $\kappa(t) = R \not= 0$. Dann ist $c$ Teil eines Kreisbogens mit Radius $\tfrac{1}{|R|}$.
\end{satz}

\begin{defn}
  Für $c : I \to \R^2$ regulär, nicht notwendigerweiße nach BL parametrisiert, ist die Krümmung zur Zeit $t$ definiert als
  \[ \frac{\det(c'(t), c''(t))}{ \| c'(t) \|^3 } \]
\end{defn}

\begin{bem}
  Obige Definition ist invariant unter orientierungserhaltenden Umparametrisierungen, und stimmt für nach BL parametrisierte Kurven mit der vorhergehenden Definition überein.
\end{bem}

\begin{satz}[Hauptsatz der lokalen ebenen Kurventheorie]
  Sei $\kappa : I \to \R$ eine stetige Funktion und $t_0 \in I$ und $x_0, v_0 \in \R^2$ mit $\| v_0 \| = 1$. Dann gibt es ganu eine nach BL parametrisierte zweimal stetig differenzierbare Kurve $c : I \to \R^2$ mit Krümmung $\kappa$, $c(t_0) = x_0$ und $c'(t_0) = v(t_0) = v_0$.
\end{satz}

\begin{defn}
  Eine reguläre Kurve $c : [a, b] \to \R^n$ heißt \emph{geschlossen}, falls $c(a) = c(b)$ und $c'(a) = c'(b)$.
  Eine reguläre geschlossene Kurve $c$ heißt \emph{einfach geschlossen}, wenn $c|_{[a, b[}$ injektiv ist.
\end{defn}

\begin{defn}
  Für eine geschlossene reguläre ebene Kurve $c : [a, b] \to \R^2$ heißt die Zahl
  \[ \overline{\kappa}(c) \coloneqq \Intabdt{\kappa(t) \| c'(t) \|} \]
  \emph{Totalkrümmung} von $c$.
\end{defn}

\begin{bem}
  Ist $c$ nach BL parametrisiert, so ist $\overline{\kappa}(c) = \Intabdt{\kappa(t)}$.
\end{bem}

\begin{satz}
  Die Totalkrümmung ist invariant unter orientierungserhaltenden Umparametrisierungen, d.\,h. ist $c : [a_2, b_2] \to \R^2$ eine reguläre Kurve und $\phi : [a_1, b_1] \to [a_2, b_2]$ eine Diffeomorphismus mit $\phi' > 0$, dann gilt $\overline\kappa(c) = \overline\kappa(c \circ \phi)$.
\end{satz}

\begin{satz}[\emph{Polarwinkelfunktion}]
  Sei $\gamma = \begin{psmallmatrix} \gamma_1 \\ \gamma_2 \end{psmallmatrix} : [a, b] \to S^1$ stetig (glatt) und $\omega_a \in \R$, sodass $\gamma(a) = e^{i \omega_a}$. Dann gibt es eine eindeutige stetige (glatte) Abbildung $\omega : [a, b] \to \R$, genannt Polarwinkelfunktion von $\gamma$ mit $\omega(a) = \omega_a$ und $\gamma(t) = e^{i \omega(t)} = \begin{pmatrix} \cos(\omega(t)) \\ \sin(\omega(t)) \end{pmatrix}$ für alle $t \in [a, b]$.
\end{satz}

\begin{satz}
  Seien $\omega$ und $\tilde\omega$ zwei stetige Polarwinkelfunktionen zu einer stetigen Abbildung $\gamma : [a, b] \to S^1$. Dann gibt es ein $k \in \Z$, sodass $\omega(t) - \tilde\omega(t) = 2 \pi k$ für alle $t \in [a, b]$.
\end{satz}

\begin{satz}
  Sei $c : [a, b] \to \R^2$ eine ebene reguläre geschlossene Kurve, dann heißt die ganze Zahl
  \[ U_c \coloneqq \frac{1}{2 \pi} \overline\kappa(c) = \frac{1}{2 \pi} \Intabdt{\kappa(t) \| c'(t) \|} \]
  \emph{Tangentendrehzahl} oder \emph{Umlaufzahl} von $c$.
\end{satz}

\begin{satz}[Umlaufsatz von Hopf]
  Die Tangentendrehzahl einer einfach geschlossenen regulären Kurve ist $\pm 1$.
\end{satz}

\begin{satz}
  Für die Absolutkrümmung einer einfach geschlossenen regulären Kurve $c : [a, b] \to \R^2$ gilt $\kappa_{\text{abs}} \geq 2 \pi$, wobei Gleichheit genau dann gilt, wenn $\kappa_c$ das Vorzeichen nicht wechselt.
\end{satz}

\begin{satz}[Whitney-Graustein]
  Für zwei glatte reguläre geschlossene ebene Kurven $c, d : [0, 1] \to \R^2$ sind folgende Aussagen äquivalent:
  \begin{multicols}{2}
    \begin{enumerate}[label=(\roman*),leftmargin=2em]
      \item $c$ ist zu $d$ regulär homotop
      \item $U_c = U_d$
    \end{enumerate}
  \end{multicols}
\end{satz}

\begin{defn}
  Eine glatte reguläre Kurve $c : I \to \R^n$ ($n \geq 3$) heißt \emph{Frenet-Kurve}, wenn für alle $t \in I$ die Ableitungen $c'(t), c''(t), ..., c^{(n-1)}(t)$ linear unabhängig sind.
\end{defn}

\begin{defn}
  Sei $c : I \to \R^n$ eine Frenet-Kurve und $t \in I$. Wende das Gram-Schmidtsche Orthogonalisierungsverfahren auf $\{ c'(t), c''(t), ..., c^{(n-1)}(t) \}$ an und ergänze das resultierende Orthonormalsystem $(b_1(t), ..., b_{n-1}(t))$ mit einem passenden Vektor $b_n(t)$ zu einer Orthonormalbasis, die positiv orientiert ist. Die so definierten Funktionen $b_1, ..., b_n : I \to \R^n$ sind stetig und werden zusammen das \emph{Frenet-$n$-Bein} von $c$ genannt.
\end{defn}

\begin{defn}
  Sei $(b_1, ..., b_n)$ das Frenet-$n$-Bein einer Frenet-Kurve $c$. Dann gilt:
  \[ A \coloneqq (\langle b_j' , b_k \rangle)_{jk} = \begin{pmatrix}
    0 & \kappa_1 &&& 0 \\
    - \kappa_1 & 0 & \kappa_2 \\
    & \ddots & \ddots & \ddots \\
    && - \kappa_{n-2} & 0 & \kappa_{n-1} \\
    0 &&& - \kappa_{n-1} & 0
  \end{pmatrix} \]
  Die Funktion $\kappa_j : I \to \R, t \mapsto \langle b_j'(t) , b_{j+1}(t) \rangle, j = 1, ..., n - 1$ heißt $j-te$ \emph{Frenet-Krümmung} von $c$.
\end{defn}

% Aufgabe 1.23

\begin{satz}[Hauptsatz der lokalen Raumkurventheorie]
  Seien $\kappa_1, ..., \kappa_{n-1} : I \to \R$ glatte Funktionen mit $\kappa_1, ..., \kappa_{n-2} > 0$ und $t_0 \in I$ und $\{ v_1, ..., v_n \}$ eine positiv orientierte Orthonormalbasis, sowie $x_0 \in \R^n$. Dann gibt es genau eine nach BL parametrisierte Frenet-Kurve $c : I \to \R^n$, sodass gilt
  \begin{itemize}
    \item $c(t_0) = x_0$,
    \item das Frenet-$n$-Bein von $c$ in $t_0$ ist $\{ v_1, ..., v_n \}$ und
    \item die $j$-te Frenet-Krümmung von $c$ ist $\kappa_j$.
  \end{itemize}
\end{satz}

\begin{defn}[Frenet-Kurven im $\R^3$]
  Sei $c : I \to \R^3$ eine nach BL parametrisierte Frenet-Kurve und $t \in I$. Dann heißt
  \begin{itemize}
    \item $b_1(t) = v(t) = c'(t)$ der \emph{Tangentenvektor} an $c$ in $t$,
    \item $b_2(t) = \tfrac{c''(t)}{\| c''(t) \|}$ \emph{Normalenvektor} an $c$ in $t$,
    \item $\mathrm{span}(b_1(t), b_2(t))$ \emph{Schmiegebene} an $c$ in $t$,
    \item $b_3(t) = b_1(t) \times b_2(t)$ \emph{Binormalenvektor} an $c$ in $t$,
    \item $\tau_c(t) = \tau(t) \coloneqq \kappa_2(t) = \langle b_2'(t) , b_3(t) \rangle$ \emph{Torsion} o. \emph{Windung} von~$c$.
  \end{itemize}
\end{defn}

\begin{bem}
  Die Frenet-Gleichungen für nach BL parametrisierte Frenet-Kurven im $\R^3$ lauten
  \[
      b_1' = \kappa_2 b_2, \quad
      b_2' = - \kappa_c b_1 + \tau_c b_3, \quad
      b_3' = - \tau_c b_2
  \]
\end{bem}

% Nach Theorem 1.24 bestimmt die Kru ̈mmung und die Torsion eine nach Bogenla ̈nge
% parametrisierte Frenet-Kurve im R3 bis auf eine euklidische Bewegung eindeutig

\begin{bem}
  Für eine nicht nach BL parametrisierte Frenet-Kurve $c : I \to \R^3$ gilt für Krümmung und Torsion
  \[ \kappa_c \coloneqq \frac{\| c' \times c'' \|}{\| c' \|^3} \quad \text{und} \quad \tau_c \coloneqq \frac{\det(c', c'', c''')}{\| c' \times c'' \|^2}. \]
\end{bem}

% TODO: Redundant?

\begin{defn}
  Für eine glatte geschlossene reguläre Kurve $c : [a, b] \to \R^n$ ist die \emph{Totalkrümmung} definiert durch
  \[ \overline\kappa(c) \coloneqq \Intabdt{\kappa_c(t) \cdot \| c'(t) \|}. \]

  Hierbei ist die Krümmung einer regulären Raumkurve $c : I \to \R^n$ wie folgt definiert:
  Sei $\phi : I \to J$ orientierungserhaltend
  (d.\,h. $\phi' > 0$) und so gewählt,
  dass $\tilde{c} \coloneqq c \circ \phi^{-1} : J \to \R^n$ nach
  BL parametrisiert ist, dann definieren wir $\kappa_c(t) \coloneqq \kappa_{\tilde{c}}(\phi(t))$.
\end{defn}

\begin{satz}[Fenchel]
  Für eine geschlossene reguläre glatte (oder $\mathcal{C}^2$) Kurve $c : [a, b] \to \R^3$ gilt
  \[ \overline\kappa(c) \geq 2 \pi. \]
  Gleichheit tritt genau dann ein, wenn $c$ eine einfach geschlossene konvexe reguläre glatte (oder $\mathcal{C}^2$) Kurve ist, die in einer affinen Ebene des $\R^3$ liegt.
\end{satz}

\begin{satz}
  Sei $v : [0, b] \to S^2 \subset \R^3$ eine stetige rektifizierbare Kurve der Länge $L < 2 \pi$
mit $c(0) = c(b)$, so liegt das Bild von $v$ ganz in einer offenen Hemisphäre.
\end{satz}


\section{Lokale Flächentheorie}

\begin{nota}
  Sei im Folgenden $m \in \N$ und $U \subset \R^m$ offen.
\end{nota}

\begin{defn}
  Sei $f : U \to \R^n$ eine Abbildung und $v \in \R^m \setminus \{ 0 \}$. Dann heißt
  \[ \partial_v f(u) \coloneqq \lim_{h \to 0} \frac{f(u + hv) - f(u)}{ h } \]
  \emph{Richtungsableitung} von $f$ im Punkt $u$ (falls der Limes existiert). Für $v = e_j$ heißt
  \[ \partial_j f(u) \coloneqq \partial_{e_j} f(u) \]
  \emph{partielle Ableitung} nach der $j$-ten Variable. Falls die partielle Ableitung für alle $u \in U$ existiert, erhalten wir eine Funktion $\partial_j : U \to \R^n, u \mapsto \partial_j f(u)$. Definiere
  \[ \partial_{j_1, j_2, ..., j_k} f \coloneqq \partial_{j_1} ( \partial_{j_2} ( ... ( \partial_{j_k} f) ) ). \]
\end{defn}

\begin{defn}
  Eine Abbildung $f : U \to \R^n$ heißt $\mathcal{C}^k$-Abbildung, wenn alle $k$-ten partiellen Ableitungen von $f$ existieren und stetig sind. Wenn $f \in \mathcal{C}^k$ für beliebiges $k \in \N$, so heißt $f$ \emph{glatt}.
\end{defn}

\begin{satz}[Schwarz]
  Ist $f$ eine $\mathcal{C}^k$-Abbildung, so kommt es bei allen $l$-ten partiellen Ableitungen mit $l \leq k$ nicht auf die Reihenfolge der partiellen Ableitungen an.
\end{satz}

\begin{defn}
  Eine Abbildung $f : U \to \R^n$ heißt in $u \in U$ \emph{total differenzierbar}, wenn gilt: Es gibt eine lineare Abbildung $D_u f = \partial f_u : \R^m \to \R^n$, genannt das \emph{totale Differential} von $f$ in $u$, sodass für genügend kleine $h \in \R^n$ gilt:
  \[ f(u + h) = f(u) + \partial f_u(h) + o(h) \]
  für eine in einer Umgebung von $0$ definierten Funktion $o : \R^n \to \R^m$ mit $\lim_{h \to 0} \tfrac{o(h)}{\| h \|} = 0$.
\end{defn}

\begin{defn}
  Für eine total differenzierbare Funktion $f$ heißt die Matrix $J_u f = (D_u f(e_1), ..., D_u f(e_n))$ \emph{Jacobi-Matrix} von $f$ in $u$.
\end{defn}

\begin{bem}
Es gelten folgende Implikationen:\\
$\quad\quad\,\,\, f$ ist stetig partiell differenzierbar\\
$\implies$ $f$ ist total differenzierbar ($\implies f$ ist stetig)\\
$\implies$ $f$ ist partiell differenzierbar
\end{bem}

% TODO: Kettenregel

\begin{defn}
  Eine total differenzierbare Abbildung $f : U \to \R^n$ heißt regulär oder Immersion, wenn für alle $u \in U$ gilt: $\mathrm{Rang}(J_u f) = m$, d.\,h. alle partiellen Ableitungen sind in jedem Punkt linear unabhängig und $J_u f$ ist injektiv. Insbesondere muss $m \leq n$ gelten.
\end{defn}

\begin{defn}
  Sei $X : U \to \R^n$ eine (glatte) Immersion. Dann heißt das Bild $f(U)$ \emph{immergierte Fläche}, immersierte Fläche oder parametrisiertes Flächenstück. Sei $\tilde{U}$ offen in $\R^n$ und $\phi : \tilde{U} \to U$ ein Diffeomorphismus, dann heißt $\tilde{X} \coloneqq X \circ \phi : \tilde{U} \to \R^n$ \emph{Umparametrisierung} von $X$.
\end{defn}

\begin{nota}
  Sei im folgenden $X : U \to \R^n$ eine Immersion.
\end{nota}

\begin{defn}
  Für $u \in U$ heißt der Untervektorraum
  \[ T_u X \coloneqq \mathrm{span}(\partial_1 X(u), ..., \partial_m X(u)) = \mathrm{Bild}(D_u X) \subset \R^n \]
  \emph{Tangentialraum} von $X$ in $u$ und sein orthogonales Komplement $N_u X \coloneqq (T_u X)^\perp \subset \R^n$ \emph{Normalraum} an $X$ in $u$.
\end{defn}

\begin{bem}
  Für $u \in U$ definiert
  \[ \langle v, w \rangle_u \coloneqq \langle D_u X(v), D_u X(w) \rangle_{\mathrm{eukl}} \]
  ein Skalarprodukt auf dem $\R^m$. Die Positiv-Definitheit folgt dabei aus der Injektivität von $D_u$.
\end{bem}

\begin{bem}
  Bezeichne mit $\mathrm{SymBil}(\R^m)$ die Menge der symmetrischen Bilinearformen auf $\R^m$.
\end{bem}

\begin{defn}
  Die \emph{erste Fundamentalform} (FF) einer Immersion $X$ ist die Abbildung
  \[ \I : U \to \mathrm{SymBil(\R^m)}, \quad u \mapsto \I_u \coloneqq \langle \cdot , \cdot \rangle_u. \]
  Äquivalent dazu wird auch die Abbildung
  \[ g : U \to \R^{m \times m}, \quad u \mapsto g_u \coloneqq (J_u X)^T (J_u X) \]
  manchmal als erste Fundamentalform bezeichnet.
\end{defn}

\begin{defn}
  Sei $c : [a, b] \to \R^n$ eine glatte Kurve. Wir nennen $c$ eine \emph{Kurve auf X}, wenn es eine glatte Kurve $\alpha : [a, b] \to U$ gibt, sodass $c = X \circ \alpha$
\end{defn}

\begin{bem}
  Im obigen Fall gilt
  \[ L(c) \coloneqq \Intabdt{\|c'(t)\|} = \Intabdt{\| D_{\alpha(t)} X(\alpha'(t)) \|}. \]
\end{bem}

\begin{bem}
  Seien $c_1 = X \circ \alpha_1$ und $c_2 = X \circ \alpha_2$ zwei reguläre Kurven auf $X$, die sich in einem Punkt schneiden, d.\,h. $\alpha_1(t_1) = \alpha_2(t_2) =: u$. Dann ist der Schnittwinkel $\measuredangle(c_1'(t), c_2'(t))$ von $c_1$ und $c_2$ in $X(u)$ gegeben durch:
  \begin{align*}
    \cos(\measuredangle(c_1'(t), c_2'(t))) &= \frac{\langle c_1'(t_1) , c_2'(t_2) \rangle}{\| c_1'(t_1) \| \cdot \| c_2'(t_2) \|} \\
    &= \frac{I_u(\alpha_1'(t_1), \alpha_2'(t_2))}{\sqrt{I_u(\alpha_1'(t_1), \alpha_1'(t_1)) \cdot I_u(\alpha_2'(t_2), \alpha_2'(t_2))}}
  \end{align*}
\end{bem}

\begin{defn}
  Sei $C \subset U$ eine kompakte messbare Teilmenge, dann heißt
  \[ A(X(C)) \coloneqq \Int{C}{}{ \sqrt{\det(g_u)} }{u} \]
  der Flächeninhalt von $X(C)$.
\end{defn}

\begin{satz}[Transformation der ersten FF]
  Sei $\tilde{X} = X \circ \phi$ eine Umparametrisierung von $X$ mit einem Diffeo $\phi : \tilde{U} \to U$, dann gilt für $\tilde{g}_{\tilde{u}} = (J_{\tilde{u}} \tilde{X})^T (J_{\tilde{u}} \tilde{X})$:
  \[ \tilde{g}_{\tilde{u}} = (J_{\tilde{u}}(\phi))^T \cdot g_{\phi(\tilde{u})} \cdot J_{\tilde{u}}(\phi). \]
\end{satz}

\begin{bsp}[Drehfläche]
  Sei $c : I \to \R_{> 0} \times \R, t \mapsto (r(t), z(t))$ eine reguläre glatte Kurve. Dann heißt
  \[ X : I \times \R \to \R^3, \quad (t, s) \mapsto (r(t) \cos(s), r(t) \sin(s), z(t)) \]
  \emph{Drehfläche} mit Profilkurve $c$. Es gilt:
  \[ g_{(t, s)} = \begin{pmatrix} \| c'(t) \|^2 & 0 \\ 0 & r(t)^2 \end{pmatrix} \]
\end{bsp}

\begin{bsp}[Kugelfläche] Die Einheitssphäre im $\R^3$ ist
\[ X : \R^2 \to \R^3, \quad (s, t) \mapsto (- \sin(t) cos(t), \cos^2(t), \sin(t)). \]
\end{bsp}

\begin{defn}
  Zwei Immersionen $X : U \to \R^n$ und $\tilde{X} : \tilde{U} \to \R^k$ heißen \emph{lokal isometrisch}, wenn es eine Umparametrisierung $\phi : U \to \tilde{U}$ gibt, sodass die ersten Fundamentalformen von $X$ und $\tilde{X} \circ \phi$ übereinstimmen. Ist eine Immersion $X$ isometrisch zu einer Immersion, deren Bild eine offene Teilmenge einer affinen Ebene ist, so heißt $X$ \emph{abwickelbar}.
\end{defn}

% Beispiel: Zylinderfläche
% Beispiel: flacher Torus
% Beispiel: Kegelfläche
% Beispiel: Katenoid und Wendelfläche
% Beispiel: Regelfläche

\begin{defn}
  Sei $X : U \to \R^n$ eine Immersion mit $U \subset \R^{n-1}$ offen. Dann heißt $X$ \emph{Hyperfläche} im $\R^n$.
\end{defn}

\begin{bem}
  Es gilt in diesem Fall offenbar $\dim T_u = n - 1$ und $\dim N_u = 1$ für $u \in U$ und für einen Vektor $\nu_u \in N_u X \setminus \{ 0 \}$ gilt $N_u X = \R \cdot v_u$.
\end{bem}

\begin{defn}
  $v_u \coloneqq \sum_{j=1}^{n} \det(\partial_1 X(u), ..., \partial_{n-1} X(u), e_j) e_j$
\end{defn}

\begin{bem}
  Es gilt:
  \begin{itemize}
    \item $v_u \in N_u X \setminus \{ 0 \}$
    \item $\det(\partial_1 X(u), ..., \partial_{n-1} X(u), v_u) > 0$
  \end{itemize}
\end{bem}

\begin{bem}
  Für $n = 3$ und $m = 2$ gilt $v_u = \partial_1 X(u) \times \partial_2 X(u)$.
\end{bem}

\begin{defn}
  Für eine Hyperfläche $X : U \to \R^n$ heißt
  \[ \nu : U \to S^{n-1} = \Set{ x \in \R^n }{ \| x \| = 1 }, \quad u \mapsto \nu_u \coloneqq \tfrac{v_u}{\| v_u \|} \]
  \emph{Gaußabbildung}.
\end{defn}

\begin{satz}
  Die Gaußabbildung einer Hyperfläche ist invariant unter orientierungserhaltenden Umparametrisierungen, d.\,h. ist $\phi : \tilde{U} \to U$ ein Diffeo mit $\det(J_{\tilde{u}} \phi) > 0$ für alle $\tilde{u} \in \tilde{U}$, dann ist $\tilde{\nu} = \nu \circ \phi$.
\end{satz}

\begin{nota}
  $\Bil(\R^m, \R^n) \coloneqq \Set{ B : \R^m \times \R^m \to \R^n }{ B \text{ bilinear } }$
\end{nota}

\begin{defn}
  Die \emph{vektorwertige zweite Fundamentalform} ist die Abbildung einer Immersion $X$ ist die Abbildung
  \begin{align*}
    \II : U \to \Bil(\R^m, \R^n), \quad & u \mapsto \II(u) = \II_u, \text { mit } \\
    \II_u : \R^m \times \R^m \to \R^n, \quad & (v, w) \mapsto \II_u(v, w) \coloneqq (\partial_v \partial_w X(u))^{N_u},
  \end{align*}
  wobei $(\cdot)^{N_u}$ die orthogonale Projektion auf den Normalenraum bezeichnet.
\end{defn}

\begin{bem}
  Nach dem Satz von Armandus Schwarz ist $\II_u$ eine symmetrische Bilinearform.
\end{bem}

\begin{bem}
  Für eine Hyperfläche $X : U \to \R^n, \, (U \opn \R^{n-1})$ gilt
  \[ \II_u(v, w) = h_u(v, w) \nu_u \quad \text{ mit } \quad h_u(v, w) = \langle \II_u(v, w) , \nu_u \rangle. \]
\end{bem}

\begin{defn}
  Die Abbildung
  \[ h : U \to \SymBil(\R^{n-1}), u \mapsto h_u = h(u) \]
  mit $h_u(v, w) = \langle \II_u(v, w), \nu_u \rangle = \langle \partial_v \partial_w X(u), \nu_u \rangle$ heißt \emph{zweite Fundamentalform} der Hyperfläche $X$.
\end{defn}

\begin{bem}
  Man kann die zweite FF auch als matrixwertige Abbildung
  \[ h : U \to \R^{(n-1) \times (n-1)}, \quad u \mapsto (h_{jk}(u)) = \langle \partial_j \partial_k X(u), \nu_u \rangle \]
  aufassen.
\end{bem}

\begin{satz}
  Für die Gaussabbildung $\nu$ einer Hyperfläche $X : U \to \R^n$ gilt für alle $j, k \in \{ 1, ..., m \}$
  \[ \langle \partial_j \nu , \partial_k X \langle = - h_{jk} \quad \text{und} \quad \langle \partial_j \nu, \nu \rangle = 0. \]
\end{satz}

\begin{defn}
  Sei $X : U \to \R^n$ eine Hyperfläche und $u \in U$, dann heißt die lineare Abbildung
  \[ W_u \coloneqq - D_u \nu \circ (D_u X)^{-1} : T_u X \to T_u X \]
  \emph{Weingartenabbildung} von $X$ im Punkt $u$.
\end{defn}

\begin{bem}
  Es gilt $W_u(\partial_j X(u)) = - \partial_j \nu(u)$.
\end{bem}

\begin{satz}
  \begin{itemize}
    \item $W_u$ ist selbstadjungiert bzgl. der Einschränkung $\langle \cdot , \cdot \rangle_{T_u}$.
    \item $h_{jk}(u) = \langle W_u(\partial_j X(u)), \partial_k X(u) \rangle$
    \item Die Weingartenabbildung ist invariant unter orientierungserhaltenden Umparametrisierungen, d.\,h. ist $\phi : \tilde{U} \to U$ ein Diffeo mit $\det(J\phi) > 0$, dann gilt für $\tilde{X} \coloneqq X \circ \phi$ und alle $\tilde{u} \in \tilde{U}$: $W_{\phi(\tilde{u})} = \tilde{W}_{\tilde{u}}$.
  \end{itemize}
\end{satz}

\begin{satz}
  Sei $g_u = (g_{jk}(u))$ die Matrix der ersten und $h_u = (h_{jk}(u))$ die Matrix der zweiten FF einer Hyperfläche $X$, dann gilt für die Matrix $w_u = (w_{jk}(u))$ von $W_u$ bzgl. der Basis $\{ \partial_1 X(u), ..., \partial_{n-1} X(u) \}$ von $T_u X$:
  \[ w_u = g_u^{-1} \cdot h_u \]
\end{satz}

\begin{bem}
  Die Weingartenabbildung ist als selbstadjungierter Endo reell diagonalisierbar (Spektralsatz).
\end{bem}

\begin{defn}
  Sei $X : U \to \R^n$ eine Hyperfläche.
  \begin{itemize}
    \item Die Eigenwerte $\kappa_1(u), ..., \kappa_{n-1}(u)$ mit Vielfachheiten von $W_u$ heißen \emph{Hauptkrümmungen} von $X$ in $u$ und die dazugehörigen Eigenvektoren \emph{Hauptkrümmungsrichtungen} von $X$ in $u$.
    \item Die \emph{mittlere Krümmung} von $X$ ist definiert als
    \[ H : U \to \R, \quad u \mapsto \tfrac{1}{n-1} \, \spur(W_u) = \tfrac{1}{n-1} \sum_{j=1}^{n-1} \kappa_j(u). \]
    \item Die \emph{Gauß-(Kronecker-)Krümmung} von $X$ ist die Abbildung
    \[ K : U \to \R, \quad u \mapsto \det(W_u) = \frac{\det(h_u)}{\det{g_u}} = \prod_{j=1}^{n-1} \kappa_j(u). \]
  \end{itemize}
\end{defn}

\begin{satz}
  Die Hauptkrümmungen, die mittlere Krümmung und die Gauß-Kronecker-Krümmung sind invariant unter orientierungserhaltenden Umparametrisiserungen.
\end{satz}

% Beispiel: Drehfläche

\begin{satz}
  Sei $X : U \to \R^n$ eine Hyperfläche und $u_0 \in U$ ein Punkt. Dann gibt es eine offene Umgebung $U_0 \opn U$ von $u_0$ und eine Umparametrisierung $\phi : U_0 \to \tilde{U}$, sodass für $\tilde{X} \coloneqq X \circ \phi^{-1}$ gilt:

  Es gibt eine glatte (bzw. $\mathcal{C}^2$) Funktion $f : \tilde{U} \to \R$ mit $D_{\phi(u_0)} f = 0$, sodass $\tilde{X} = \Graph(f)$, d.\,h. es gilt für alle $\tilde{u} \in \tilde{U}$:
  \[ \tilde{X}(\tilde{u}) = (\tilde{u}, f(\tilde{u})). \]
\end{satz}

\begin{nota}
  $\nabla f = (\partial_1 f, ..., \partial_k f)$ heißt \emph{Gradient} von $f : \R^k \to \R^m$.
\end{nota}

\begin{satz}
  Sei $U \opn \R^{n-1}$ und $f : U \to \R$ glatt. Dann ist die zweite FF der Graphen-Hyperfläche $X : U \to \R^n, u \mapsto (u, f(n))$
  \[ h_{jk}(u) = \frac{\partial_{jk} f(u)}{\sqrt{1 + |\nabla f(u)|^2}}. \]
\end{satz}

\begin{satz}
  Sei $X : U \to \R^n$ eine Hyperfläche, $u_0 \in U$, sowie $E_{u_0} \coloneqq X(u_0) + T_{u_0} X$ die affine Tangentialebene an $X$ in $u_0$. Dann gilt:
  \begin{itemize}
    \item Ist $K(u_0) > 0$, so liegt für eine kleine offene Umgebung $U_0 \subset U$ von $u_0$ das Bild $X(U_0)$ ganz auf einer Seite von $E_{u_0}$.
    \item Ist $K(u_0) < 0$, so trifft für jede Umgebung $U_0 \subset U$ von $u_0$ das Bild $X(U_0)$ beide Seiten von $E_{u_0}$.
  \end{itemize}
\end{satz}

\end{document}
