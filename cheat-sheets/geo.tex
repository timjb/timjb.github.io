\documentclass[a4paper,10pt,landscape]{article}
\usepackage{multicol}
\usepackage{calc}
\usepackage{ifthen}
\usepackage[landscape]{geometry}
\usepackage{amsmath,amsthm,amsfonts,amssymb}
\usepackage{color,graphicx,overpic}
\usepackage{hyperref}
\usepackage[utf8]{inputenc}
\usepackage[ngerman]{babel}
\usepackage{enumitem}
\usepackage{mathtools}
\usepackage{pifont}

\setitemize[0]{leftmargin=10pt,itemindent=0pt,itemsep=0pt}
\setenumerate[0]{leftmargin=10pt,itemindent=0pt,itemsep=0pt}

\newcommand{\cmark}{\ding{51}}
\newcommand{\xmark}{\ding{55}}
\newcommand{\R}{\mathbb{R}}
\newcommand{\N}{\mathbb{N}}
\newcommand{\Z}{\mathbb{Z}}
\newcommand{\C}{\mathbb{C}}
\newcommand{\Q}{\mathbb{Q}}
\newcommand{\PS}{\mathcal{P}} % Powerset
\newcommand{\PSO}{\PS(\Omega)} % Powerset
\newcommand{\Alg}{\mathfrak{A}}
\newcommand{\Ring}{\mathfrak{R}}

% Differentiator
\renewcommand{\d}{\mathrm{d}}

\pdfinfo{
  /Title (geo3.pdf)
  /Creator (TeX)
  /Producer (pdfTeX 1.40.0)
  /Author (Tim Baumann)
  /Subject (Geometrie Zusammenfassung)
  /Keywords (geometry,overview)}

% This sets page margins to .5 inch if using letter paper, and to 1cm
% if using A4 paper. (This probably isn't strictly necessary.)
% If using another size paper, use default 1cm margins.
\ifthenelse{\lengthtest { \paperwidth = 11in}}
    { \geometry{top=.5in,left=.5in,right=.5in,bottom=.5in} }
    {\ifthenelse{ \lengthtest{ \paperwidth = 297mm}}
        {\geometry{top=1cm,left=1cm,right=1cm,bottom=1cm} }
        {\geometry{top=1cm,left=1cm,right=1cm,bottom=1cm} }
    }

\theoremstyle{definition}

\newtheorem*{nota}{Notation}
\newtheorem*{defn}{Definition}
\newtheorem*{prob}{Problem}
\newtheorem*{bsp}{Beispiel}
\newtheorem*{satz}{Satz}
\newtheorem*{kor}{Korollar}
\newtheorem*{acht}{Achtung}
\newtheorem*{strat}{Strategie}

\theoremstyle{remark}
\newtheorem*{bem}{Bemerkung}

% Römische Ziffern
\makeatletter
\newcommand*{\rom}[1]{\expandafter\@slowromancap\romannumeral #1@}
\makeatother

% Ober- und Unterintegral, siehe
% http://tex.stackexchange.com/questions/44237/lower-and-upper-riemann-integrals
\def\upint{\mathchoice%
    {\mkern13mu\overline{\vphantom{\intop}\mkern7mu}\mkern-20mu}%
    {\mkern7mu\overline{\vphantom{\intop}\mkern7mu}\mkern-14mu}%
    {\mkern7mu\overline{\vphantom{\intop}\mkern7mu}\mkern-14mu}%
    {\mkern7mu\overline{\vphantom{\intop}\mkern7mu}\mkern-14mu}%
  \int}
\def\lowint{\mkern3mu\underline{\vphantom{\intop}\mkern7mu}\mkern-10mu\int}

% Färbe \emph{}
\definecolor{Emph}{rgb}{0.2,0.2,0.8}  %softer red for display
\renewcommand{\emph}[1]{\textcolor{Emph}{\bf{#1}}}

% Display style überall!
\everymath{\displaystyle}

% Turn off header and footer
\pagestyle{empty}

% Redefine section commands to use less space
\makeatletter
\renewcommand{\section}{\@startsection{section}{1}{0mm}%
                                {-1ex plus -.5ex minus 2ex}%
                                {2.5ex plus .2ex}%x
                                {\normalfont\large\bfseries}}
\renewcommand{\subsection}{\@startsection{subsection}{2}{0mm}%
                                {-1explus -.5ex minus -.2ex}%
                                {0.5ex plus .2ex}%
                                {\normalfont\normalsize\bfseries}}
\renewcommand{\subsubsection}{\@startsection{subsubsection}{3}{0mm}%
                                {-1ex plus -.5ex minus -.2ex}%
                                {1ex plus .2ex}%
                                {\normalfont\small\bfseries}}
\makeatother

% Don't print section numbers
\setcounter{secnumdepth}{0}

\DeclarePairedDelimiterX\Set[2]{\lbrace}{\rbrace}%
 { #1 \,\delimsize|\, #2 }

\setlength{\parindent}{0pt}
\setlength{\parskip}{0pt plus 10ex}

% -----------------------------------------------------------------------

\begin{document}
\raggedright
\footnotesize
\begin{multicols}{3}

% multicol parameters
% These lengths are set only within the two main columns
%\setlength{\columnseprule}{0.25pt}
\setlength{\premulticols}{1pt}
\setlength{\postmulticols}{1pt}
\setlength{\multicolsep}{1pt}
\setlength{\columnsep}{2pt}

\begin{center}
  \Large{\underline{Zusammenfassung Geometrie}} \\
\end{center}


\section{Geometrie von Kurven}

% Erinnerung: Skalarprodukt, Isometrie, Winkel, Orthogonalität

\begin{nota}
  Sei im Folgenden $I$ ein Intervall, d.\,h. eine zusammenhängende Teilmenge von $\R$.
\end{nota}

\begin{defn}
  Eine Abbildung $c : I \to \R^n$ heißt \emph{reguläre Kurve}, wenn $c$ beliebig oft differenzierbar ist und $c'(t) \not= 0$ für alle $t \in I$ gilt.

  Der affine Unterraum $\tau_{c,t} := c(t) + \R(c'(t))$ heißt \emph{Tangente} an $c$ im Punkt $c(t)$ bzw. Tangente an $c$ zum Zeitpunkt $t$.
\end{defn}

% Lemma 1.2: Tangenten ändern sich unter Parameterwechseln nicht

\begin{defn}
  Die \emph{Bogenlänge} (BL) einer regulären Kurve $c : [a, b] \to \R^n$ ist
  \[ L(c) := \int_a^b\! \| c'(t) \|\,\d t. \]
\end{defn}

\begin{satz}
  Die Bogenlänge ist invariant unter Umparametrisierung, d.\,h. sei $c : [a_2, b_2] \to \R^n$ eine reguläre Kurve und $\phi : [a_1, b_1] \to [a_2, b_2]$ ein Diffeomorphismus, dann gilt $L(c) = L(c \circ \phi)$.
\end{satz}

\begin{defn}
  Eine reguläre Kurve $c : I \to \R^n$ heißt \emph{nach Bogenlänge parametrisiert}, wenn $\| c'(t) \| = 1$ für alle $t \in I$.
\end{defn}

\begin{satz}
  Jede reguläre Kurve $c : I \to \R$ lässt sich nach BL parametrisieren, d.\,h. es existiert ein Intervall $J$ und ein Diffeomorphismus $\phi : J \to I$, welcher sogar orientierungserhaltend ist, sodass $\tilde{c} := c \circ \phi$ nach BL parametrisiert ist.
\end{satz}

\begin{defn}
  Zwei Vektoren $a, b \in \R^n$ heißen gleichgerichtet, falls $a = \lambda b$ für ein $\lambda \geq 0$.
\end{defn}

\begin{satz}
  Sei $v : [a, b] \to \R^n$ stetig, dann gilt
  \[ \| \int_a^b\!v(t)\,\d t \| \leq \int_a^b\!\|v(t)\|\,\d t, \]
  wobei Gleichheit genau dann gilt, falls alle $v(t)$ gleichgerichtet sind.
\end{satz}

\begin{satz}
  Sei $c : [a, b] \to \R^n$ eine reguläre Kurve und $x := c(a), y := c(b)$. Dann gilt $L(c) \geq d(x, y)$. Wenn $L(c) = d(x, y)$, dann gibt es einen Diffeomorphismus $\phi : [a, b] \to [0, 1]$, sodass
  \[ c = c_{xy} \circ \phi, \]
  wobei $c_{xy} : [0, 1] \to \R^n,\,t \mapsto x + t (y - x)$.
\end{satz}

% TODO: Definition Feinheit

\begin{defn}
  Sei $c : [a, b] \to \R^n$ eine stetige Kurve und $a = t_0 < t_1 < ... < t_k = b$ eine Zerteilung von $[a, b]$. Dann ist die Länge des \emph{Polygonzugs} durch die Punkte $c(t_i)$ gegeben durch
  \[ \hat{L}_c(t_0, ..., t_k) = \sum_{j=1}^k \| c(t_j) - c(t_{j-1}) \|. \]
\end{defn}

\begin{defn}
  Eine stetige Kurve $c$ heißt \emph{rektifizierbar} von Länge $\hat{L}_c$, wenn gilt: Für alle $\epsilon > 0$ gibt es ein $\delta > 0$, sodass für alle Unterteilungen $a = t_0 < t1 < ... < t_k = b$ der Feinheit mindestens $\delta$ gilt:
  \[ \| \hat{L}_c - \hat{L}_c(t_0, t_1, ..., t_k) \| < \epsilon. \]
\end{defn}

\begin{defn}
  Sei $c : I \to \R^n$ regulär und nach BL parametrisiert. Dann heißt der Vektor $c''(t)$ \emph{Krümmungsvektor} von $c$ in $t \in I$ und die Abbildung $\kappa : I \to \R, \quad t \mapsto \| c''(t) \|$ heißt \emph{Krümmung} der nach BL parametrisierten Kurve.
\end{defn}

\begin{defn}
  Eine Kurve $c : I \to \R^2$ wird \emph{ebene Kurve} genannt.
\end{defn}

\begin{defn}
  Sei $c$ eine reguläre, nach BL parametrisierte, ebene Kurve. Dann heißt
  \[ n = n_c : I \to \R^2, \quad t \mapsto J \cdot c'(t) \text{ mit } J := \begin{pmatrix} 0 & -1 \\ 1 & 0 \end{pmatrix}  \]
  das \emph{Normalenfeld} von $c$.
\end{defn}

\begin{bem}
  Für alle $t \in I$ bildet $(c'(t), n_c(t))$ eine positiv orientierte Orthonormalbasis des $\R^2$.
  Es gilt außerdem $c''(t) \perp c'(t)$, also $c''(t) = \kappa(t) \cdot n_c(t)$, d.\,h. die Krümmung ist im $\R^2$ vorzeichenbehaftet.
\end{bem}

\begin{satz}[\emph{Frenet-Gleichungen} ebener Kurven]
  Sei $c : I \to \R^2$ regulär, nach BL parametrisiert und $v = c'$, dann gilt
  \[ c'' = \kappa \cdot n \quad \text{ und } \quad n' = -\kappa \cdot v. \]
\end{satz}

% Matrixschreibweise?
% Beispiel: Kreis

\begin{bsp}
  Die nach BL parametrisierte gegen den UZS durchlaufene Kreislinie mit Mittelpunkt $m \in \R^2$ und Radius $r > 0$
  \[ c : \R \to \R^2, \quad t \mapsto m + r \begin{pmatrix} \cos(t/r) \\ \sin(t/r) \end{pmatrix} \]
  hat konstante Krümmung $\kappa(t) = \tfrac{1}{r}$.
\end{bsp}

\begin{satz}
  Sei $c : I \to \R^2$ glatte, nach BL parametrisiert mit konstanter Krümmung $\kappa(t) = R \not= 0$. Dann ist $c$ Teil eines Kreisbogens mit Radius $\tfrac{1}{|R|}$.
\end{satz}

\begin{defn}
  Für $c : I \to \R^2$ regulär, nicht notwendigerweiße nach BL parametrisiert, ist die Krümmung zur Zeit $t$ definiert als
  \[ \frac{\det(c'(t), c''(t))}{ \| c'(t) \|^3 } \]
\end{defn}

\begin{bem}
  Obige Definition ist invariant unter orientierungserhaltenden Umparametrisierungen, und stimmt für nach BL parametrisierte Kurven mit der vorhergehenden Definition überein.
\end{bem}

\begin{satz}[Hauptsatz der lokalen ebenen Kurventheorie]
  Sei $\kappa : I \to \R$ eine stetige Funktion und $t_0 \in I$ und $x_0, v_0 \in \R^2$ mit $\| v_0 \| = 1$. Dann gibt es ganu eine nach BL parametrisierte zweimal stetig differenzierbare Kurve $c : I \to \R^2$ mit Krümmung $\kappa$, $c(t_0) = x_0$ und $c'(t_0) = v(t_0) = v_0$.
\end{satz}

\begin{defn}
  Eine reguläre Kurve $c : [a, b] \to \R^n$ heißt \emph{geschlossen}, wenn gilt
  \begin{itemize}
    \item $c(a) = c(b)$ und
    \item $c'(a) = c'(b)$.
  \end{itemize}
  Eine reguläre geschlossene Kurve $c$ heißt \emph{einfach geschlossen}, wenn $c|_{[a, b[}$ injektiv ist.
\end{defn}

\begin{defn}
  Für eine geschlossene reguläre ebene Kurve $c : [a, b] \to \R^2$ heißt die Zahl
  \[ \overline{\kappa}(c) := \int_a^b\!\kappa(t) \| c'(t) \| \,\d t \]
  \emph{Totalkrümmung} von $c$.
\end{defn}

\begin{bem}
  Ist $c$ nach BL parametrisiert, so ist $\overline{\kappa}(c) = \int_a^b\!\kappa(t) \,\d t$.
\end{bem}

\begin{satz}
  Die Totalkrümmung ist invariant unter orientierungserhaltenden Umparametrisierungen, d.\,h. ist $c : [a_2, b_2] \to \R^2$ eine reguläre Kurve und $\phi : [a_1, b_1] \to [a_2, b_2]$ eine Diffeomorphismus mit $\phi' > 0$, dann gilt $\overline\kappa(c) = \overline\kappa(c \circ \phi)$.
\end{satz}

\begin{satz}[\emph{Polarwinkelfunktion}]
  Sei $\gamma = \begin{psmallmatrix} \gamma_1 \\ \gamma_2 \end{psmallmatrix} : [a, b] \to S^1$ stetig (glatt) und $\omega_a \in \R$, sodass $\gamma(a) = e^{i \omega_a}$. Dann gibt es eine eindeutige stetige (glatte) Abbildung $\omega : [a, b] \to \R$, genannt Polarwinkelfunktion von $\gamma$ mit $\omega(a) = \omega_a$ und $\gamma(t) = e^{i \omega(t)} = \begin{pmatrix} \cos(\omega(t)) \\ \sin(\omega(t)) \end{pmatrix}$ für alle $t \in [a, b]$.
\end{satz}

\begin{satz}
  Seien $\omega$ und $\tilde\omega$ zwei stetige Polarwinkelfunktionen zu einer stetigen Abbildung $\gamma : [a, b] \to S^1$. Dann gibt es ein $k \in \Z$, sodass $\omega(t) - \tilde\omega(t) = 2 \pi k$ für alle $t \in [a, b]$.
\end{satz}

\begin{satz}
  Sei $c : [a, b] \to \R^2$ eine ebene reguläre geschlossene Kurve, dann heißt die ganze Zahl
  \[ U_c := \frac{1}{2 \pi} \overline\kappa(c) = \frac{1}{2 \pi} \int_a^b\! \kappa(t) \| c'(t) \| \,\d t \]
  \emph{Tangentendrehzahl} oder \emph{Umlaufzahl} von $c$.
\end{satz}

\begin{satz}[Umlaufsatz von Hopf]
  Die Tangentendrehzahl einer einfach geschlossenen regulären Kurve ist $\pm 1$.
\end{satz}

\begin{satz}
  Für die Absolutkrümmung einer einfach geschlossenen regulären Kurve $c : [a, b] \to \R^2$ gilt $\kappa_{\text{abs}} \geq 2 \pi$, wobei Gleichheit genau dann gilt, wenn $\kappa_c$ das Vorzeichen nicht wechselt.
\end{satz}

\begin{satz}[Whitney-Graustein]
  Für zwei glatte reguläre geschlossene ebene Kurven $c, d : [0, 1] \to \R^2$ sind folgende Aussagen äquivalent:
  \begin{enumerate}[label=(\roman*),leftmargin=2em]
    \item $c$ ist zu $d$ regulär homotop
    \item $U_c = U_d$
  \end{enumerate}
\end{satz}

\begin{defn}
  Eine glatte reguläre Kurve $c : I \to \R^n$ ($n \geq 3$) heißt \emph{Frenet-Kurve}, wenn für alle $t \in I$ die Ableitungen $c'(t), c''(t), ..., c^{(n-1)}(t)$ linear unabhängig sind.
\end{defn}

\begin{defn}
  Sei $c : I \to \R^n$ eine Frenet-Kurve und $t \in I$. Wende das Gram-Schmidtsche Orthogonalisierungsverfahren auf $\{ c'(t), c''(t), ..., c^{(n-1)}(t) \}$ an und ergänze das resultierende Orthonormalsystem $(b_1(t), ..., b_{n-1}(t))$ mit einem passenden Vektor $b_n(t)$ zu einer Orthonormalbasis, die positiv orientiert ist. Die so definierten Funktionen $b_1, ..., b_n : I \to \R^n$ sind stetig und werden zusammen das \emph{Frenet-$n$-Bein} von $c$ genannt.
\end{defn}

\begin{defn}
  Sei $(b_1, ..., b_n)$ das Frenet-$n$-Bein einer Frenet-Kurve $c$. Dann gilt:
  \[ A := (\langle b_j' , b_k \rangle)_{jk} = \begin{pmatrix}
    0 & \kappa_1 &&& 0 \\
    - \kappa_1 & 0 & \kappa_2 \\
    & \ddots & \ddots & \ddots \\
    && - \kappa_{n-2} & 0 & \kappa_{n-1} \\
    0 &&& - \kappa_{n-1} & 0
  \end{pmatrix} \]
  Die Funktion $\kappa_j : I \to \R, t \mapsto \langle b_j'(t) , b_{j+1}(t) \rangle, j = 1, ..., n - 1$ heißt $j-te$ \emph{Frenet-Krümmung} von $c$.
\end{defn}

% Aufgabe 1.23

\begin{satz}[Hauptsatz der lokalen Raumkurventheorie]
  Seien $\kappa_1, ..., \kappa_{n-1} : I \to \R$ glatte Funktionen mit $\kappa_1, ..., \kappa_{n-2} > 0$ und $t_0 \in I$ und $\{ v_1, ..., v_n \}$ eine positiv orientierte Orthonormalbasis, sowie $x_0 \in \R^n$. Dann gibt es genau eine nach BL parametrisierte Frenet-Kurve $c : I \to \R^n$, sodass gilt
  \begin{itemize}
    \item $c(t_0) = x_0$,
    \item das Frenet-$n$-Bein von $c$ in $t_0$ ist $\{ v_1, ..., v_n \}$ und
    \item die $j$-te Frenet-Krümmung von $c$ ist $\kappa_j$.
  \end{itemize}
\end{satz}

\begin{defn}[Frenet-Kurven im $\R^3$]
  Sei $c : I \to \R^3$ eine nach BL parametrisierte Frenet-Kurve und $t \in I$. Dann heißt
  \begin{itemize}
    \item $b_1(t) = v(t) = c'(t)$ der \emph{Tangentenvektor} an $c$ in $t$,
    \item $b_2(t) = \frac{c''(t)}{\| c''(t) \|}$ \emph{Normalenvektor} an $c$ in $t$,
    \item $\mathrm{span}{b_1(t), b_2(t)}$ \emph{Schmiegebene} an $c$ in $t$,
    \item $b_3(t) = b_1(t) \times b_2(t)$ \emph{Binormalenvektor} an $c$ in $t$,
    \item $\tau_c(t) = \tau(t) := \kappa_2(t) = \langle b_2'(t) , b_3(t) \rangle$ \emph{Torsion} oder \emph{Windung} von $c$.
  \end{itemize}
\end{defn}

\begin{bem}
  Die Frenet-Gleichungen für nach BL parametrisierte Frenet-Kurven im $\R^3$ lauten
  \begin{align*}
    b_1' &= \kappa_2 b_2 \\
    b_2' &= - \kappa_c b_1 + \tau_c b_3 \\
    b_3' &= - \tau_c b_2.
  \end{align*}
\end{bem}

% Nach Theorem 1.24 bestimmt die Kru ̈mmung und die Torsion eine nach Bogenla ̈nge
% parametrisierte Frenet-Kurve im R3 bis auf eine euklidische Bewegung eindeutig

\begin{bem}
  Für eine nicht nach BL parametrisierte Frenet-Kurve $c : I \to \R^3$ gilt für Krümmung und Torsion
  \[ \kappa_c := \frac{\| c' \times c'' \|}{\| c' \|^3} \quad \text{und} \quad \tau_c := \frac{\det(c', c'', c''')}{\| c' \times c'' \|^2}. \]
\end{bem}

% TODO: Redundant?

\begin{defn}
  Für eine glatte geschlossene reguläre Kurve $c : [a, b] \to \R^n$ ist die \emph{Totalkrümmung} definiert durch
  \[ \overline\kappa(c) := \int_a^b\!\kappa_c(t) \cdot \| c'(t) \|\,\d t. \]

  Hierbei ist die Krümmung einer regulären Raumkurve $c : I \to \R^n$ wie folgt definiert:
  Sei $\phi : I \to J$ orientierungserhaltend
  (d.\,h. $\phi' > 0$) und so gewählt,
  dass $\tilde{c} := c \circ \phi^{-1} : J \to \R^n$ nach
  BL parametrisiert ist, dann definieren wir $\kappa_c(t) := \kappa_{\tilde{c}}(\phi(t))$.
\end{defn}

\begin{satz}[Fenchel]
  Für eine geschlossene reguläre glatte (oder $\mathcal{C}^2$) Kurve $c : [a, b] \to \R^3$ gilt
  \[ \overline\kappa(c) \geq 2 \pi. \]
  Gleichheit tritt genau dann ein, wenn $c$ eine einfach geschlossene konvexe reguläre glatte (oder $\mathcal{C}^2$) Kurve ist, die in einer affinen Ebene des $\R^3$ liegt.
\end{satz}

\begin{satz}
  Sei $v : [0, b] \to S^2 \subset \R^3$ eine stetige rektifizierbare Kurve der Länge $L < 2 \pi$
mit $c(0) = c(b)$, so liegt das Bild von $v$ ganz in einer offenen Hemisphäre.
\end{satz}


\section{Lokale Flächentheorie}

\begin{nota}
  Sei im Folgenden $m \in \N$ und $U \subset \R^m$ offen.
\end{nota}

\begin{defn}
  Sei $f : U \to \R^n$ eine Abbildung und $v \in \R^m \setminus \{ 0 \}$. Dann heißt
  \[ \partial_v f(u) := \lim_{h \to 0} \frac{f(u + hv) - f(u)}{ h } \]
  \emph{Richtungsableitung} von $f$ im Punkt $u$ (falls der Limes existiert). Für $v = e_j$ heißt
  \[ \partial_j f(u) := \partial_{e_j} f(u) \]
  \emph{partielle Ableitung} nach der $j$-ten Variable. Falls die partielle Ableitung für alle $u \in U$ existiert, erhalten wir eine Funktion $\partial_j : U \to \R^n, u \mapsto \partial_j f(u)$. Definiere
  \[ \partial_{j_1, j_2, ..., j_k} f := \partial_{j_1} ( \partial_{j_2} ( ... ( \partial_{j_k} f) ) ). \]
\end{defn}

\begin{defn}
  Eine Abbildung $f : U \to \R^n$ heißt $\mathcal{C}^k$-Abbildung, wenn alle $k$-ten partiellen Ableitungen von $f$ existieren und stetig sind. Wenn $f \in \mathcal{C}^k$ für beliebiges $k \in \N$, so heißt $f$ \emph{glatt}.
\end{defn}

\begin{satz}[Schwarz]
  Ist $f$ eine $\mathcal{C}^k$-Abbildung, so kommt es bei allen $l$-ten partiellen Ableitungen mit $l \leq k$ nicht auf die Reihenfolge der partiellen Ableitungen an.
\end{satz}

\begin{defn}
  Eine Abbildung $f : U \to \R^n$ heißt in $u \in U$ \emph{total differenzierbar}, wenn gilt: Es gibt eine lineare Abbildung $D_u f = \partial f_u : \R^m \to \R^n$, genannt das \emph{totale Differential} von $f$ in $u$, sodass für genügend kleine $h \in \R^n$ gilt:
  \[ f(u + h) = f(u) + \partial f_u(h) + o(h) \]
  für eine in einer Umgebung von $0$ definierten Funktion $o : \R^n \to \R^m$ mit $\lim_{h \to 0} \tfrac{o(h)}{\| h \|} = 0$.
\end{defn}

\begin{defn}
  Für eine total differenzierbare Funktion $f$ heißt die Matrix $J_u f = (D_u f(e_1), ..., D_u f(e_n))$ \emph{Jacobi-Matrix} von $f$ in $u$.
\end{defn}

\begin{bem}
Es gelten folgende Implikationen:\\
$\quad\quad\,\,\, f$ ist stetig partiell differenzierbar\\
$\implies$ $f$ ist total differenzierbar ($\implies f$ ist stetig)\\
$\implies$ $f$ ist partiell differenzierbar
\end{bem}

% TODO: Kettenregel

\begin{defn}
  Eine total differenzierbare Abbildung $f : U \to \R^n$ heißt regulär oder Immersion, wenn für alle $u \in U$ gilt: $\mathrm{Rang}(J_u f) = m$, d.\,h. alle partiellen Ableitungen sind in jedem Punkt linear unabhängig und $J_u f$ ist injektiv. Insbesondere muss $m \leq n$ gelten.
\end{defn}

\begin{defn}
  Sei $X : U \to \R^n$ eine (glatte) Immersion. Dann heißt das Bild $f(U)$ \emph{immergierte Fläche}, immersierte Fläche oder parametrisiertes Flächenstück. Sei $\tilde{U}$ offen in $\R^n$ und $\phi : \tilde{U} \to U$ ein Diffeomorphismus, dann heißt $\tilde{X} := X \circ \phi : \tilde{U} \to \R^n$ \emph{Umparametrisierung} von $X$.
\end{defn}

\begin{defn}
  Sei $f : U \to \R^n$ eine Immersion und $u \in U$. Dann heißt der Untervektorraum
  \[ T_u X := \mathrm{span}(\partial_1 X(u), ..., \partial_m X(u)) = \mathrm{Bild}(D_u X) \subset \R^n \]
  \emph{Tangentialraum} von $X$ in $u$ und sein orthogonales Komplement $N_u X := (T_u X)^\perp \subset \R^n$ \emph{Normalraum} an $X$ in $u$.
\end{defn}

\begin{bem}
  Sei $X : U \to \R^n$ eine Immersion und $u \in U$. Dann definiert
  \[ \langle v, w \rangle_u := \langle D_u X(v), D_u X(w) \rangle_{\mathrm{eukl}} \]
  ein Skalarprodukt auf dem $\R^m$. Die Positiv-Definitheit folgt dabei aus der Injektivität von $D_u$.
\end{bem}

\begin{bem}
  Bezeichne mit $\mathrm{SymBil}(\R^m)$ die Menge der symmetrischen Bilinearformen auf $\R^m$.
\end{bem}

\begin{defn}
  Die \emph{erste Fundamentalform} einer Immersion $X$ ist die Abbildung
  \[ I : U \to \mathrm{SymBild(\R^m)}, \quad u \mapsto I_u := \langle \cdot , \cdot \rangle_u. \]
  Äquivalent dazu wird auch die Abbildung
  \[ g : U \to \R^{m \times m}, \quad u \mapsto g_u := (J_u X)^T (J_u X) \]
  manchmal als erste Fundamentalform bezeichnet.
\end{defn}

\end{multicols}
\end{document}
