\documentclass{cheat-sheet}

\pdfinfo{
  /Title (Zusammenfassung Stochastik 1)
  /Author (Tim Baumann)
}

\begin{document}

\maketitle{Zusammenfassung Stochastik \rom{1}}

\section{Der abstrakte Maßbegriff}

% TODO: Ereignisalgebra?

\begin{defn}
  Eine \emph{Ereignisalgebra} oder \emph{Boolesche Algebra} ist eine Menge $\mathfrak{A}$ mit zweistelligen Verknüpfungen $\wedge$ (\glqq und\grqq) und $\vee$ (\glqq oder\grqq), einer einstelligen Verknüpfung $\overline{\,\cdot\,}$ (Komplement) und ausgezeichneten Elementen $U \in \mathfrak{A}$ (unmögliches Ereignis) und $S \in \mathfrak{A}$ (sicheres Ereignis), sodass für $A, B, C \in \mathfrak{A}$ gilt:

  \begin{multicols}{2}
    \scriptsize
    \begin{enumerate}[label=\roman*.,leftmargin=2em]
      \item $A \wedge A = A$
      \item $A \wedge B = B \wedge A$
      \item $A \wedge S = A$
      \item $A \wedge U = U$
      \item $A \wedge \overline{A} = U$
      \item $A \wedge (B \wedge C) = (A \wedge B) \wedge C$
      \item $A \vee A = A$
      \item $A \vee S = S$
      \item $A \vee U = A$
      \item $A \vee \overline{A} = S$
      \item $A \vee (B \vee C) = (A \vee B) \vee C$
      \item $A \wedge (B \vee C) = (A \wedge B) \vee (A \wedge C)$
    \end{enumerate}
  \end{multicols}
\end{defn}

\begin{defn}
  Sei $\mathfrak{A}$ eine Boolesche Algebra. Dann definiert
  \[ A \leq B \colon\iff A \wedge B = B \]
  eine Partialordnung auf $\mathfrak{A}$, gesprochen $A$ impliziert $B$.
\end{defn}

\begin{defn}
  Eine \emph{Algebra} (auch Mengenalgebra) $\mathfrak{A} \subset \mathcal{P}(\Omega)$ ist ein System von Teilmengen einer Menge $\Omega$ mit $\emptyset \in \mathfrak{A}$, das unter folgenden Operationen stabil ist:
  \begin{itemize}
    \item Vereinigung: $A, B \in \mathfrak{A} \implies A \cup B \in \mathfrak{A}$
    \item Durchschnitt: $A, B \in \mathfrak{A} \implies A \cap B \in \mathfrak{A}$
    \item Komplementbildung: $A \in \mathfrak{A} \implies A^c \coloneqq \Omega \backslash A \in \mathfrak{A}$
  \end{itemize}
\end{defn}

\begin{satz}[Isomorphiesatz von Stone]
Zu jeder Booleschen Algebra $\mathfrak{A}$ gibt es eine Menge $\Omega$ derart, dass $\mathfrak{A}$ isomorph zu einer Mengenalgebra $\mathfrak{A}$ in $\mathcal{P}(\Omega)$ ist.
\end{satz}

\begin{defn}
  Eine \emph{$\sigma$-Algebra} ist eine Algebra $\mathfrak{A} \subset \mathcal{P}(\Omega)$, die nicht nur unter endlichen, sondern sogar unter abzählbaren Vereinigungen stabil ist, d.\,h.

  \[ (A_n)_{n \in \N} \text{ Folge in } \mathfrak{A} \implies \bigcup_{n = 0}^{\infty} A_n \in \mathfrak{A}. \]
\end{defn}

\begin{bem}
  Es gilt damit:

  \begin{itemize}
    \item $\Omega = \emptyset^c \in \mathfrak{A}$
    \item Abgeschlossenheit unter abzählbaren Schnitten:
  \[ (A_n)_{n \in \N} \text{ Folge in } \mathfrak{A} \implies \bigcap_{n = 0}^{\infty} A_n = \left( \bigcup_{n = 0}^{\infty} (A_n)^c \right)^c \in \mathfrak{A}. \]
  \end{itemize}
\end{bem}

\begin{defn}
  Sei $(A_n)_{n \in \N}$ eine Folge in einer $\sigma$-Algebra $\mathfrak{A}$. Dann sind der Limes Superior und Limes Inferior der Folge $A_n$ wie folgt definiert:

  \[ \limsup_{n \to \infty} A_n \coloneqq \bigcap_{n = 1}^{\infty} \bigcup_{m = n}^{\infty} A_n \in \mathfrak{A} \]
  \[ \liminf_{n \to \infty} A_n \coloneqq \bigcup_{n = 1}^{\infty} \bigcap_{m = n}^{\infty} A_n \in \mathfrak{A} \]
\end{defn}

\begin{bem}
  In einer $\sigma$-Algebra, in der die Mengen mögliche Ereignisse beschreiben, ist der Limes Superior das Ereignis, das eintritt, wenn unendlich viele Ereignisse der Folge $A_n$ eintreten. Der Limes Infinum tritt genau dann ein, wenn alle bis auf endlich viele Ereignisse der Folge $A_n$ eintreten.
\end{bem}

\begin{defn}
  Ein \emph{Ring} $\mathfrak{A} \subset \mathcal{P}(\Omega)$ ist ein System von Teilmengen einer Menge $\Omega$ mit $\emptyset \in \mathfrak{A}$, das unter folgenden Operation stabil ist:

  \begin{itemize}
    \item Vereinigung: $A, B \in \mathfrak{A} \implies A \cup B \in \mathfrak{A}$
    \item Differenz: $A, B \in \mathfrak{A} \implies B \backslash A = B \cap A^c \in \mathfrak{A}$
  \end{itemize}

  Ein Ring, der nicht nur unter endlicher, sondern sogar unter abzählbarer Vereinigung stabil ist, heißt \emph{$\sigma$-Ring}.
\end{defn}

\begin{bem}
  $\mathfrak{A}$ ($\sigma$-)\,Algebra $\iff$ $\mathfrak{A}$ ($\sigma$-)\,Ring und $\Omega \in \mathfrak{A}$.
\end{bem}

\begin{satz}
  Sei $(\mathfrak{A}_i)_{(i \in I)}$ eine Familie von ($\sigma$-)\,Ringen / ($\sigma$-)\,Algebren über einer Menge $\Omega$. Dann ist auch $\cup_{i \in I} \mathfrak{A}_i$ ein ($\sigma$-)\,Ring / eine ($\sigma$-)\,Algebra über $\Omega$.
\end{satz}

%\begin{defn}
%  Sei $$
%\end{defn}

\end{document}
