\documentclass[a4paper,10pt,landscape]{article}
\usepackage{multicol}
\usepackage{calc}
\usepackage{ifthen}
\usepackage[landscape]{geometry}
\usepackage{amsmath,amsthm,amsfonts,amssymb}
\usepackage{color,graphicx,overpic}
\usepackage{hyperref}
\usepackage[utf8]{inputenc}
\usepackage[ngerman]{babel}
\usepackage{enumitem}
\usepackage{mathtools}
\usepackage{pifont}

\setitemize[0]{leftmargin=10pt,itemindent=0pt,itemsep=0pt}
\setenumerate[0]{leftmargin=10pt,itemindent=0pt,itemsep=0pt}

\newcommand{\cmark}{\ding{51}}
\newcommand{\xmark}{\ding{55}}
\newcommand{\R}{\mathbb{R}}
\newcommand{\N}{\mathbb{N}}
\newcommand{\Z}{\mathbb{Z}}
\newcommand{\C}{\mathbb{C}}

% Differentiator
\renewcommand{\d}{\mathrm{d}}

\pdfinfo{
  /Title (stochi1.pdf)
  /Creator (TeX)
  /Producer (pdfTeX 1.40.0)
  /Author (Tim Baumann)
  /Subject (Stochastik 1 Zusammenfassung)
  /Keywords (overview)}

% This sets page margins to .5 inch if using letter paper, and to 1cm
% if using A4 paper. (This probably isn't strictly necessary.)
% If using another size paper, use default 1cm margins.
\ifthenelse{\lengthtest { \paperwidth = 11in}}
    { \geometry{top=.5in,left=.5in,right=.5in,bottom=.5in} }
    {\ifthenelse{ \lengthtest{ \paperwidth = 297mm}}
        {\geometry{top=1cm,left=1cm,right=1cm,bottom=1cm} }
        {\geometry{top=1cm,left=1cm,right=1cm,bottom=1cm} }
    }

\theoremstyle{definition}

\newtheorem*{nota}{Notation}
\newtheorem*{defn}{Definition}
\newtheorem*{bsp}{Beispiel}
\newtheorem*{satz}{Satz}
\newtheorem*{kor}{Korollar}
\newtheorem*{acht}{Achtung}
\newtheorem*{strat}{Strategie}

\theoremstyle{remark}
\newtheorem*{bem}{Bemerkung}

% Römische Ziffern
\makeatletter
\newcommand*{\rom}[1]{\expandafter\@slowromancap\romannumeral #1@}
\makeatother

% Ober- und Unterintegral, siehe
% http://tex.stackexchange.com/questions/44237/lower-and-upper-riemann-integrals
\def\upint{\mathchoice%
    {\mkern13mu\overline{\vphantom{\intop}\mkern7mu}\mkern-20mu}%
    {\mkern7mu\overline{\vphantom{\intop}\mkern7mu}\mkern-14mu}%
    {\mkern7mu\overline{\vphantom{\intop}\mkern7mu}\mkern-14mu}%
    {\mkern7mu\overline{\vphantom{\intop}\mkern7mu}\mkern-14mu}%
  \int}
\def\lowint{\mkern3mu\underline{\vphantom{\intop}\mkern7mu}\mkern-10mu\int}

% Offene und abgeschlossene Teilmengen
% http://tex.stackexchange.com/questions/22371/subseteq-circ-as-a-single-symbol-open-subset
\newcommand\opn{\mathrel{\ooalign{$\subset$\cr
  \hidewidth\raise.1ex\hbox{$\circ\mkern.5mu$}\cr}}}
\newcommand\cls{\mathrel{\ooalign{$\subset$\cr
  \hidewidth\raise.1ex\hbox{$\bullet\mkern.5mu$}\cr}}}

% Färbe \emph{}
\definecolor{Emph}{rgb}{0.2,0.2,0.8}  %softer red for display
\renewcommand{\emph}[1]{\textcolor{Emph}{\bf{#1}}}

% Display style überall!
\everymath{\displaystyle}

% Turn off header and footer
\pagestyle{empty}

% Redefine section commands to use less space
\makeatletter
\renewcommand{\section}{\@startsection{section}{1}{0mm}%
                                {-1ex plus -.5ex minus 2ex}%
                                {2.5ex plus .2ex}%x
                                {\normalfont\large\bfseries}}
\renewcommand{\subsection}{\@startsection{subsection}{2}{0mm}%
                                {-1explus -.5ex minus -.2ex}%
                                {0.5ex plus .2ex}%
                                {\normalfont\normalsize\bfseries}}
\renewcommand{\subsubsection}{\@startsection{subsubsection}{3}{0mm}%
                                {-1ex plus -.5ex minus -.2ex}%
                                {1ex plus .2ex}%
                                {\normalfont\small\bfseries}}
\makeatother

% Don't print section numbers
\setcounter{secnumdepth}{0}

\DeclarePairedDelimiterX\Set[2]{\lbrace}{\rbrace}%
 { #1 \,\delimsize|\, #2 }

\setlength{\parindent}{0pt}
\setlength{\parskip}{0pt plus 10ex}

% -----------------------------------------------------------------------

\begin{document}
\raggedright
\footnotesize
\begin{multicols}{3}

% multicol parameters
% These lengths are set only within the two main columns
%\setlength{\columnseprule}{0.25pt}
\setlength{\premulticols}{1pt}
\setlength{\postmulticols}{1pt}
\setlength{\multicolsep}{1pt}
\setlength{\columnsep}{2pt}

\begin{center}
  \Large{\underline{Zusammenfassung Stochastik \rom{1}}} \\
\end{center}

\section{Der abstrakte Maßbegriff}

% TODO: Ereignisalgebra?

\begin{defn}
  Eine \emph{Ereignisalgebra} oder \emph{Boolesche Algebra} ist eine Menge $\mathfrak{A}$ mit zweistelligen Verknüpfungen $\wedge$ (\glqq und\grqq) und $\vee$ (\glqq oder\grqq), einer einstelligen Verknüpfung $\overline{\,\cdot\,}$ (Komplement) und ausgezeichneten Elementen $U \in \mathfrak{A}$ (unmögliches Ereignis) und $S \in \mathfrak{A}$ (sicheres Ereignis), sodass für $A, B, C \in \mathfrak{A}$ gilt:

  \begin{multicols}{2}
    \scriptsize
    \begin{enumerate}[label=\roman*.,leftmargin=2em]
      \item $A \wedge A = A$
      \item $A \wedge B = B \wedge A$
      \item $A \wedge S = A$
      \item $A \wedge U = U$
      \item $A \wedge \overline{A} = U$
      \item $A \wedge (B \wedge C) = (A \wedge B) \wedge C$
      \item $A \vee A = A$
      \item $A \vee S = S$
      \item $A \vee U = A$
      \item $A \vee \overline{A} = S$
      \item $A \vee (B \vee C) = (A \vee B) \vee C$
      \item $A \wedge (B \vee C) = (A \wedge B) \vee (A \wedge C)$
    \end{enumerate}
  \end{multicols}
\end{defn}

\begin{defn}
  Sei $\mathfrak{A}$ eine Boolesche Algebra. Dann definiert
  \[ A \lte B :\iff A \wegde B = B \]
  eine Partialordnung auf $\mathfrak{A}$, gesprochen $A$ impliziert $B$.
\end{defn}

\begin{defn}
  Eine \emph{Algebra} (auch Mengenalgebra) $\mathfrak{A} \subset \mathcal{P}(\Omega)$ ist ein System von Teilmengen einer Menge $\Omega$ mit $\emptyset \in \mathfrak{A}$, das unter folgenden Operationen stabil ist:
  \begin{itemize}
    \item Vereinigung: $A, B \in \mathfrak{A} \implies A \cup B \in \mathfrak{A}$
    \item Durchschnitt: $A, B \in \mathfrak{A} \implies A \cap B \in \mathfrak{A}$
    \item Komplementbildung: $A \in \mathfrak{A} \implies A^c := \Omega \backslash A \in \mathfrak{A}$
  \end{itemize}
\end{defn}

\begin{satz}[Isomorphiesatz von Stone]
Zu jeder Booleschen Algebra $\mathfrak{A}$ gibt es eine Menge $\Omega$ derart, dass $\mathfrak{A}$ isomorph zu einer Mengenalgebra $\mathfrak{A}$ in $\mathcal{P}(\Omega)$ ist.
\end{satz}

\begin{defn}
  Eine \emph{$\sigma$-Algebra} ist eine Algebra $\mathfrak{A} \subset \mathcal{P}(\Omega)$, die nicht nur unter endlichen, sondern sogar unter abzählbaren Vereinigungen stabil ist, d.\,h.

  \[ (A_n)_{n \in \N} \text{ Folge in } \mathfrak{A} \implies \bigcup_{n = 0}^{\infty} A_n \in \mathfrak{A}. \]
\end{defn}

\begin{bem}
  Es gilt damit:

  \begin{itemize}
    \item $\Omega \in \mathfrak{A}$
    \item Abgeschlossenheit unter abzählbaren Schnitten:
  \[ (A_n)_{n \in \N} \text{ Folge in } \mathfrak{A} \implies \bigcap_{n = 0}^{\infty} A_n = \left( \bigcup_{n = 0}^{\infty} (A_n)^c \right)^c \in \mathfrak{A}. \]
  \end{itemize}
\end{bem}

\begin{defn}
  Sei $(A_n)_{n \in \N}$ eine Folge in einer $\sigma$-Algebra $\mathfrak{A}$. Dann sind der Limes Superior und Limes Inferior der Folge $A_n$ wie folgt definiert:

  \[ \limsup_{n \to \infty} A_n := \bigcap_{n = 1}^{\infty} \bigcup_{m = n}^{\infty} A_n \in \mathfrak{A} \]
  \[ \liminf_{n \to \infty} A_n := \bigcup_{n = 1}^{\infty} \bigcap_{m = n}^{\infty} A_n \in \mathfrak{A} \]
\end{defn}

\begin{bem}
  In einer $\sigma$-Algebra, in der die Mengen mögliche Ereignisse beschreiben, ist der Limes Superior das Ereignis, das eintritt, wenn unendlich viele Ereignisse der Folge $A_n$ eintreten. Der Limes Infinum tritt genau dann ein, wenn alle bis auf endlich viele Ereignisse der Folge $A_n$ eintreten.
\end{bem}

\begin{defn}
  Ein \emph{Ring} $\mathfrak{A} \subset \mathcal{P}(\Omega)$ ist ein System von Teilmengen einer Menge $\Omega$ mit $\emptyset \in \mathfrak{A}$, das unter folgenden Operation stabil ist:

  \begin{itemize}
    \item Vereinigung: $A, B \in \mathfrak{A} \implies A \cup B \in \mathfrak{A}$
    \item Differenz: $A, B \in \mathfrak{A} \implies B \backslash A = B \cap A_C \in \mathfrak{A}$
  \end{itemize}

  Ein Ring, der nicht nur unter endlicher, sondern sogar unter abzählbarer Vereinigung stabil ist, heißt \emph{$\sigma$-Ring}.
\end{defn}

\begin{bem}
  $\mathfrak{A}$ ($\sigma$-)\,Algebra $\iff$ $\mathfrak{A}$ ($\sigma$-)\,Ring und $\Omega \in \mathfrak{A}$.
\end{bem}

\end{multicols}
\end{document}
